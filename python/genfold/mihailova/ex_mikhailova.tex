\documentclass[a4paper,12pt]{article}
\usepackage{amsmath,amsfonts,amsthm,amscd,amssymb,latexsym}%,eufrak}
%%%%%%%%%%%%%
\usepackage{caption}
\usepackage{subcaption}
\usepackage{enumerate,graphicx,psfrag}%,subfigure}%,jchangebar,oldgerm}
\usepackage[mathscr]{eucal}
\usepackage[usenames]{color}
\usepackage{url}
\usepackage[shortlabels]{enumitem}
\usepackage{comment}
%\usepackage[utf8]{inputenc}
\usepackage[T1]{fontenc}
%\usepackage{showkeys}
\usepackage{wrapfig}
\usepackage{lscape}
\usepackage{rotating}
%%%%%%%%%
\sloppy
%%%%%%%%%%%%%%%%%%%%%%%
\title{A subgroup}

\renewcommand{\a}{\alpha }
\renewcommand{\b}{\beta }
\newcommand{\G}{\Gamma }
\newcommand{\g}{\gamma }
\newcommand{\D}{\Delta }
\renewcommand{\d}{\delta }
%\def\vd{\vardelta}
\newcommand{\ep}{\epsilon }
\newcommand{\e}{\varepsilon }
\newcommand{\z}{\zeta }
%\eta
\renewcommand{\th}{\theta }
\newcommand{\T}{\Theta }
\renewcommand{\i}{\iota }
\renewcommand{\k}{\kappa }
\renewcommand{\l}{\lambda }
\renewcommand{\L}{\Lambda }
%\mu
%\nu
%\xi
%omicron
%\pi
\renewcommand{\r}{\rho }
\newcommand{\s}{\sigma }
\renewcommand{\S}{\Sigma }
\renewcommand{\t}{\tau }
\newcommand{\up}{\upsilon }
\newcommand{\U}{\Upsilon }
%\phi
\newcommand{\x}{\chi }
%\psi
\newcommand{\W}{\Omega }
\newcommand{\w}{\omega }
%%%%%%%%%%%%%%%%%%%%%%%%%%%%%%%
%%%%%%%%%%%%%%%%%%%%%%%%%%%%%
\newcommand{\pd}{\partial}
\newcommand{\wht}{\widehat}
%\newcommand{\cC}{{\mathcal C}}
%\newcommand{\cdim}{\texttt{cdim}}
\newcommand{\fC}{{\textswab C}}
\newenvironment{ef}{\noindent\color{blue} \bf EF: }{}
%
\newcommand{\cA}{{\cal{A}}}
\newcommand{\cD}{{\cal{D}}}
\newcommand{\cF}{{\cal{F}}}
\newcommand{\cH}{{\cal{H}}}
\newcommand{\cJ}{{\cal{J}}}
\newcommand{\cK}{{\cal{K}}}
\newcommand{\cP}{{\cal{P}}}
\newcommand{\cQ}{{\cal{Q}}}
\newcommand{\cR}{{\cal{R}}}
\newcommand{\cS}{{\cal{S}}}
\newcommand{\cV}{{\cal{V}}}
\newcommand{\cW}{{\cal{W}}}
%\newcommand{\GG}{\ensuremath{\mathbb{G}}}
\newcommand{\pp}{\mathbf{p}}
%%%%%%%%%%%%%%%%%%%%%%%%%%%%%%
\newcommand{\nul}{\emptyset }
\newcommand{\vim}{\nu\textrm{-im}}
%%%%%%%%%%%%%%%%%%%%%%%%%%%%%%
\newtheorem{theorem}{Theorem}[section]
\newtheorem{lemma}[theorem]{Lemma}
\newtheorem{corollary}[theorem]{Corollary}
\newtheorem{proposition}[theorem]{Proposition}
\newtheorem{axiom}[theorem]{Axiom}
\newtheorem{definition}[theorem]{Definition}
\newtheorem*{defn*}{Definition}
\newtheorem{conjecture}[theorem]{Conjecture}
%cvs -d :pserver:najd2@cvs.mas.ncl.ac.uk:/CVS/najd2
\newtheorem{exam}[theorem]{Example}
%\newtheorem{comment}[theorem]{Comment}
%
%
\newenvironment{example}{\begin{exam} \rm}{\end{exam}}
%
%
%
\newtheorem{remk}[theorem]{Remark}
\newenvironment{remark}{\begin{remk} \rm}{\end{remk}}
%
%%%%%%%%%%%%
\numberwithin{equation}{section}
\numberwithin{figure}{section}
%%%%%%%%%%%%%%%%%%%%
\newcommand{\Loop}{\operatorname{Loop}}
\newcommand{\Iso}{\operatorname{Isom}}
\newcommand{\Aut}{\operatorname{Aut}}
%%%%%%%%%%%%%%%%%%%
\renewcommand{\AA}{\ensuremath{\mathbb{A}}}
\newcommand{\ZZ}{\ensuremath{\mathbb{Z}}}
\newcommand{\QQ}{\ensuremath{\mathbb{Q}}}
\newcommand{\RR}{\ensuremath{\mathbb{R}}}
\newcommand{\NN}{\ensuremath{\mathbb{N}}}
\newcommand{\CC}{\ensuremath{\mathbb{C}}}
\newcommand{\FF}{\ensuremath{\mathbb{F}}}
%\renewcommand{\ker}{\verb"Ker"}
\newcommand{\cC}{\mathcal{C}}
\renewcommand{\cF}{\mathcal{F}}
\newcommand{\cO}{\mathcal{O}}
\renewcommand{\cS}{\mathcal{S}}
\newcommand{\la}{\langle}
\newcommand{\ra}{\rangle}
%\newcommand{\BA}{\ensuremath{\mathbb{A}}}
%%%%%%%%%%%%%%%%%%%%%%%%%%%%%%%%%%%%%%
\newcommand{\maps}{\rightarrow}
\newcommand{\ov}[1]{\overline{#1}}
\newcommand{\bs}{\backslash}
%%%%%%%%%%%%%%%%%%%%%%%%%%%%%%%
\newcommand{\be}{\begin{enumerate}}
\newcommand{\ee}{\end{enumerate}}
\newcommand{\bd}{\begin{description}}
\newcommand{\ed}{\end{description}}
\newcommand{\biz}{\begin{itemize}}
\newcommand{\eiz}{\end{itemize}}
%%%%%%%%%%%%%%%%%%%%%%%%%%%%%%%%%%%
%
\newenvironment{ajd1}{\noindent\color{red} AJD }{}
\newcommand{\ajd}[1]{\begin{ajd1} #1 \end{ajd1}}
%
%\includecomment{comp}% to see environment comp
\excludecomment{comp}% to hide environment comp
%
\begin{document}

Here I give a sketch of an example where the algorithm solving the MP does not work a priori.


Let $F_1$ be the free group on generators $x_1,x_2, t_1, t_2$,
$H_1$ be the subgroup of $F_1$ generated by 
$h_1= t_1^{-1} x_1 t_1$, 
$h_2 = t_1^{-1} x_2 t_1$, 
$h_3 =x_1$, 
$h_4 = x_2$,
$h_5= t_2 x_1 t_2^{-1}$, 
$h_6 = t_2 x_2 t_2^{-1}$.
Further, let $F_2$ be the free group on generators $y_1, y_2, s_1, s_2$,
$H_2$ be the subgroup of $F_2$ generated by 
$h'_1= y_1$, 
$h'_2 = y_2$, 
$h'_3 = s_1^{-1} y_1 s_1 $, 
$h'_4 = s_1^{-1} y_2 s_1$,
$h'_5= s_2^{-1} y_1 s_2$, 
$h'_6 = s_2 y_2 s_2$.
Consider the amalgamated product $G$ of $F_1, F_2$ over $H_1=H_2$ (with the isomorphism $h_i = h'_i$, $i = 1,\ldots 6$).

{\bf Exercise.} {\it There exists a subgroup $\mathbb{F}^2$ of $G$ such that $\mathbb{F}^2$ is isomorphic to $F_2 \times F_2$.
(hint: $\mathbb{F}^2$ is generated by $t_1s_1, t_1s_2t_2, x_1, x_2$).}

\ajd{Is there a marking scheme? I don't want to use all my time on this if it's not worth many marks.}

Let $H$ be a finitely presented $2-$generated group with unsolvable Word Problem; generated by $x_1, x_2$, for example. If it's presentation satisfies some technical conditions, then there exists a rescursively presented (moreover, finitely generated) group $M(H)$, the Mikhailova subgroup of $\mathbb{F}^2$ with unsolvable MP.
If I'm not mistaken, we don't need to compute relators of $M(H)$. So we have a subgroup in $\mathbb{F}^2$ which is in turn a subgroup in $G$ with unsolvable MP.

The missing technical part is $H$. The smallest group with unsolvable WP I know is due to Collins, and it is $10-$generated. Therefore, there are two ways: either we are trying to find $2-$generated $H$ or slightly changing the example above.
You can read on Mikhailova's subgroup in her original paper in russian or in paper by Bogopolsky and Ventura, if needed. 
The group $\mathbb{F}^2$, however, came from different side.

Suppose $H$ is a group
with presentation $\la x_1, \ldots x_m\,|\, r_1,\ldots ,r_n\ra$ and unsolvable
word problem. Let $w_1, \ldots , w_m$ be elements of the free group on
2 generators, $\FF(a,b)$, which freely generate a subgroup of rank $m$. 
Let $\phi:\FF(x_1,\ldots, x_m) \maps \FF(a,b)$ 
be the homomorphism such that $\phi(x_i)=w_i$, 
$i=1,\ldots, m$. 
Let $R_j=\phi(r_j)$, $j=1,\ldots ,n$ and let $N$ be the normal closure of 
the $R_j$'s in $\FF(a,b)$ and let $G=\FF(a,b)/N$, a group with presentation
$\la a, b|R_1, \ldots, R_n\ra$. 
Then $\phi$ induces a homomorphism $\bar\phi:H\maps  G$. 
In fact, $\bar\phi$ is an injection (as $\phi$ maps $\FF(x_1,\ldots ,x_{m})$, 
isomorphically onto it's image) so $H\le G$. The latter therefore has
unsolvable word problem. Of course, if we start with $m=10$, say, then 
 the generators $w_i$ may not be too nice,
and the relators $R_i$ may be quite horrible, but if this works we could 
make this computation.

\ajd{New bit starts here.\\} 

To input the above into the main program I'll denote by $D$ 
the subgroup of $G$ which is isomorphic to  
$\FF^2$, so 
\[D\cong \la t_1s_1,t_1s_2t_2\ra\times \la x_1,x_2\ra=\FF_2\times \FF_2,\]
a direct product of two free groups of rank 2. The isomorphism from
$\FF^2$ to $D$ takes $(u,v)$ to $uv$. Then, as in your email, 
\[\FF^2\ge M(H)=\la (t_1s_1,x_1), (t_1s_2t_2,x_2), (1,R_i), 0\le i \le 26\ra.\]


As the main programme writes the groups $F_1$ and $F_2$ using generators 
$x_i$ and $y_i$, and the relators $R_i$ in generators $x_1,x_2$: 
 I'm going to change notation, so that $t_1=x_3$, $t_2=x_4$, and  
 $s_1=y_3'$ and $s_2=y_4$. This means
that \\
$\FF_1= \FF(x_1,x_2,x_3, x_4)$,
$H_1$ is the subgroup of $F_1$ generated by 
$h_1= x_3^{-1} x_1 x_3$, 
$h_2 = x_3^{-1} x_2 x_3$, 
$h_3 =x_1$, 
$h_4 = x_2$,
$h_5= x_4 x_1 x_4^{-1}$, 
$h_6 = x_4 x_2 x_4^{-1}$.
Further, let $F_2$ be the free group on generators $y_1, y_2, y_3, y_4$,
$H_2$ be the subgroup of $F_2$ generated by 
$h'_1= y_1$, 
$h'_2 = y_2$, 
$h'_3 = y_3^{-1} y_1 y_3 $, 
$h'_4 = y_3^{-1} y_2 y_3$,
$h'_5= y_4^{-1} y_1 y_4$, 
$h'_6 = y_4^{-1} y_2 y_4$.
(My $h'_6$ is different from yours.)
As before $G$ is the amalgamated product  of $F_1, F_2$ over $H_1=H_2$ (with the isomorphism $h_i = h'_i$, $i = 1,\ldots 6$).

The image $K$ of $M(H)$ in $G$ is then generated by 
the images of $ (x_3y_3,x_1), (x_3y_4x_4,x_2), (1,R_i)$ in $G$, namely 
\[\{x_3y_3x_1, x_3y_4x_4x_2, R_i, 0\le i \le 26\}\]
and this is the subgroup I'm inputting as $K$ to the main program.
\end{document}
