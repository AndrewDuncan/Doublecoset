If $q\in Q$ and there are paths in $\G_A$ from a vertex of $\cS$ to $q$
and from $q$ to a vertex of $\cF$ then $q$ is said to be \emph{trim}.
If all vertices $q\in Q$ are trim then $A$ is called \emph{trim}.
 {\ajd For an involutive, rooted automaton $A$, being trim is equivalent
to $\G_A$ being connected. Surely it would be easier to use
connected instead of trim?}
 {\ef Definitely! Should we remove this definition at all then?}
{\ajd Let's use ``connected'' and just mention somewhere that this
is equivalent to trim in our case.}

%%%%%%%%%%%%%%%%%%
%%%%%%%%%%%%%%%%%

\begin{itemize}
\item If $\nu(\theta^{-1}(u))\cap \nu(\theta^{-1}(v))\neq \nul$ then
identify vertices $u$ and $v$.
\item If $(u,a,v)$ and $(u^\prime, a, v^\prime)$ are edges
of $\D^\prime_5$ such that $u$ is identified with $u^\prime$ and
$v$ is identified with $v^\prime$, in $\D^{\prime\prime}$, then
identify  $(u,a,v)$ and $(u^\prime, a, v^\prime)$.
%\item If $x\in \nu(\theta^{-1}(u))$ and $y\in \nu(\theta^{-1}(v))$ and
%there exists an edge $(x,a,y)$ in $\D$ then add an edge $(u,a,v)$ to
%$\D^{\prime\prime}$.
\end{itemize}
The root vertex of $\D^{\prime\prime}$ is the image, in
$\D^{\prime\prime}$, of $\nu^{-1}(1)$, where $1$ is  the root of
$\D$.
For $u \in V(\D^\prime_5)$ and $(u,a,v) \in E(\D^\prime_5)$ let
$[u]$ and $[(u,a,v)]$ denote equivalence classes in
$\D^{\prime\prime}$ with respect to this identification of
vertices and edges respectively.

\begin{lemma}\label{lem:resol-quot}
Let $[\cdot]$ be the equivalence relation in $\D^\prime_5$
described above. There is a morphism $\rho$ from  $\D$ to
$\D^{\prime\prime}$ given by $\rho(z)=[\theta(\nu^{-1}(z))]$, for
$z\in V(\D)\cup E(\D)$.
\end{lemma}
\begin{proof}
First $\rho$ must be shown to be well defined. If $z \in V(\D)\cup
E(\D)$ then either $z\in V(\D_Z)\cup E(\D_Z)$ or $z$ belongs to
some $X_k$ component of $\D$. Hence
 $\nu^{-1}(x)\neq \nul$. If $u$ and $v\in \D^\prime$ such that $u, v \in \nu^{-1}(z)$ then, by definition of
$\D^{\prime\prime}$, $\theta(u)$ and $\theta(v)$ are equivalent.
Hence $\rho$ is well-defined.

To show that $\rho$ is a morphism, suppose that $(u,a,x)$ is an edge
of $\D$. If $e=(u,a,v)$ is an edge of a component of $\D^\prime$ then
either $e$ is an edge of an $X_k$ component of $\D$ or
 $(u,a,v)\in E_Z=E(\D_Z)$. In either  case
$\rho(u,a,v)=[(\theta(u),a,\theta(v))]$ is an edge of $\D^{\prime\prime}$.
\end{proof}



\begin{comment}
\begin{lemma}\label{lem:resol-quot}
For $u\in V(\D_5)$ and $(u,a,v) \in E(\D_5)$ let $[u]$ and
$[(u,a,v)]$ denote equivalence classes in $\D^{\prime\prime}$.
There is a morphism $\rho$ from  $\D$ to $\D^{\prime\prime}$ given
by $\rho(z)=[\theta(\nu^{-1}(z))]$, for $z\in V(\D)\cup E(\D)$.
\end{lemma}
\begin{proof}
First $\rho$ must be shown to be well defined. If $x\in V(\D)\cup E(\D)$ then
either $z\in V(\D_Z)\cup E(\D_Z)$ or $z$ belongs to  some $X_k$ component of
$\D$. Hence
 $\nu^{-1}(x)\neq \nul$. If
$u$ and $v\in \nu^{-1}(z)$ then, by definition of $\D^{\prime\prime}$,
$u$ and $v$ are identified. Hence $\rho$ is well-defined.

To show that $\rho$ is a morphism, suppose that $(u,a,x)$ is an edge
of $\D$. If $e=(u,a,v)$ is an edge of a component of $\D^\prime$ then
either $e$ is an edge of an $X_k$ component of $\D$ or
 $(u,a,v)\in E_Z=E(\D_Z)$. In either  case
$\rho(u,a,v)=[(\theta(u),a,\theta(v))]$ is an edge of $\D^{\prime\prime}$.
\end{proof}
\end{comment}