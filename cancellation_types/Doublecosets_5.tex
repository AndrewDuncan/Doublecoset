\documentclass[11pt]{article}

\usepackage{amsmath,amsthm,amsfonts,amssymb}
\usepackage{enumerate,graphicx,psfrag,subfigure}
\usepackage[usenames]{color}
\usepackage{hyperref}
\usepackage{soul}

\textwidth=125mm
\textheight=195mm

\newtheorem{theorem}{Theorem}[section]
\newtheorem{lemma}[theorem]{Lemma}
\theoremstyle{definition}
\newtheorem{corollary}[theorem]{Corollary}
\newtheorem{definition}[theorem]{Definition}
\newtheorem{proposition}[theorem]{Proposition}
%\newtheorem{example}[theorem]{Example}
\newtheorem{remark}[theorem]{Remark}

\def\b{{\beta}}
\def\A{{\mathcal{A}}}
\def\a{{\alpha}}
\def\d{{\delta}}
\def\g{{\gamma}}

\newcommand{\fracd}[2]{{\displaystyle\frac{#1}{#2}}}



\title{Double cosets in free groups}
\author{Elizaveta Frenkel \\ Vladimir N. Remeslennikov}


\begin{document}
\maketitle

%%\address[E.~Frenkel]{Moscow State University,
%%GSP-1, Leninskie gory, 119991, Moscow, Russia}
%%\email{lizzy.frenkel@gmail.com}
%%
%%\address[V.~N.~Remeslennikov]{Omsk Branch of
%%Mathematical Institute SB RAS, 13 Pevtsova Street, Omsk 644099,
%%Russia} \email{remesl@iitam.omsk.net.ru}

\noindent \textsf{Elizaveta Frenkel, Moscow State University,
GSP-1, Leninskie gory, 119991, Moscow, Russia}

\noindent {\tt lizzy.frenkel@gmail.com}

\medskip

\noindent \textsf{Vladimir N. Remeslennikov, Omsk Branch of
Mathematical Institute SB RAS, 13 Pevtsova Street, Omsk 644099,
Russia}

\noindent {\tt remesl@iitam.omsk.net.ru}

\begin{abstract} In this paper we study double cosets of finite rank free groups.
We focus our attention on cancellation types in double cosets and
their formal language properties with an eye on complexity of
algorithmic problems in free products with amalgamation and HNN
extensions of groups.
\end{abstract}

MSC2010: 20E05, 20F10, 20F65

Key words: double cosets, subgroups of free groups, regular
subsets, bounded cancellations.


\section{Introduction}

By classical Nielsen-Schreier theorem every subgroup of a free
group is again free. One of the goals of this paper is to study
how one finitely generated free group $C$ can be embedded into
another free group $F$. For a fixed embedding of $C$ in $F$
naturally appears such complexes as left and right cosets, and
double cosets of type $CfC$, $f \in F$. A description of
combinatorial and formal language properties of these complexes is
the second goal of our work.

Similar questions have been studied widely as auxiliary ones, for
example, in papers \cite{BeFe}, \cite{khm}, \cite{MO} in relation
to hyperbolicity of amalgamated products of groups, and in
\cite{CPAI}, \cite{fmrI}, \cite{fmrII} in connection with decision
of the conjugacy problem in amalgamated products.

There exist a lot of papers about the characterization of
embeddings of a subgroup $C$ in a group $F$, but we mention only
the most close to the subject of our paper. In \cite{BaMR} was
shown that the property of a finitely generated subgroup $C$ to be
malnormal is algorithmically decidable in a free group $F$,
meanwhile \cite{BriWi} provide an example of undecidability of
this problem in a class of hyperbolic groups. Another related
result was proved in \cite{Jitsukawa}, where was shown that a
random finitely generated subgroup of a free group $F$ is
malnormal.

The essential role in previously mentioned studies play Cayley
graphs of a group, Schreier graphs for a subgroup $C$ in $F$ and
geodesic triangles and quadrangles in a Cayley graph of a group,
several sides of which represent elements of $C$. The present
paper lay down these results systematically and obtain several new
results, among which should be mentioned the following theorems:

{\bf Theorem \ref{th:bounded}.} Let $F$ be a free group with a
finite bases $X$ and $C$ be a finitely generated subgroup of $F$
with a Nielsen set of generators $Y$. Suppose $f \in F
\smallsetminus C$ and $f$ is of minimal length in a double coset
$CfC$. Then there exists a natural number $k$ which can be
effectively determined by $C$ and $Y$ such that

\begin{itemize} \item[1.] if $C_f = C \bigcap f^{-1} C f = 1$, then
the multiplication in complex $CfC$ is $k-$reduced, which means
$l_X(c_1) + l_X(f) + l_X(c_2) - l_X(c_1 f c_2) \le 2 k$, where
$c_1, c_2 \in C$ and $l_X(\cdot)$ is the length of elements in
$F(X)$.

\item[2.] if $C_f \neq 1$, then $CfT$ is $k-$reduced, where $T$ is
a relative Schreier transversal of $C$ in $F$ relative to $C_f$.
\end{itemize}


{\bf Theorem \ref{th:cfcreg}.} Let $C$ be a finite rank subgroup
of a free group $F(X)$. Then the set of all reduced words
representing elements of $CfC$ is regular in $F(X)$.


\section{Preliminaries}\label{Section:preliminaries}


In this section, following \cite{eps,fmrI,km}, we give some basics
on regular subsets, cosets, and Schreier representatives of
finitely generated subgroups of a finite rank free group $F$. We
will use the following notation throughout the paper.

Let $F = F(X)$ be a free group with basis $X=\{x_1,\dots, x_m\}$.
We identify  elements of $F$ with  reduced words  in the alphabet
$X \cup X^{-1}$ (i.e. \emph{$X-$reduced } words), and let $l_X(f)$
denote the length of an element $f \in F$, i.e. the number of
letters of the reduced word $f$ in $X \cup X^{-1}$.

Fix a subgroup $C = \langle h_1, \ldots, h_r \rangle$ of $F$
generated by finitely many elements $h_1, \ldots, h_r  \in F$ and
an arbitrary element $f \in F$.

The set $CfC = \{c_1 f c_2 | c_1, c_2 \in C \}$ is called a
\emph{double coset} and $f$ is a \emph{representative} of this
coset.

\subsection{Regular subsets in a free group and finite state automata}\label{subs_reg}

A {\em non-deterministic finite state automaton} $\A$ is a
quintuple $(S(\A),X,\d,S_0,F_0)$, where $S(\A)$ is a finite set of
{\em states}, $X$ is a finite set called the {\em alphabet}, $\d$
is a map $\d:S(\A) \times (X \cup \varepsilon) \mapsto S(\A)$,
called the {\em transition function}, $S_0 \subset S(\A)$ is the
(non-empty) set of {\em initial states} and $F_0 \subseteq S(\A)$
is the set of {\em final states}. Here $\varepsilon$ is assumed
not to lie in $X$.

An {\em arrow} in $\A$ is a triple $(s_1,x,s_2)$, where $s_1, s_2$
are elements of $S(\A)$ and $x$ is an element of $X \cup
\varepsilon$ and is called the {\em label} of the arrow. The {\em
source} of the arrow is $s_1$ and its {\em target} is $s_2$ .

By a path of arrows in $\A$ we mean a sequence $(s_1, u_1, s_2,
\ldots, u_n, s_{n+1})$, where $n \ge 0$ and each $u_i$ is an arrow
with source $s_i$ and target $s_{i+1}$, for $1 \le i \le n$. We
call $s_1$ the source and $s_{n+1}$ the target of the path of
arrows. Let $w_i$ be the label of $u_i$, i.e. a letter of $X$ or
$\varepsilon$. Let $w$ be the concatenation $w_1 \cdots w_n$; if
$n = 0$, set $w = \varepsilon$. Then $w$ is called the {\em label}
of the path. Let $X^{\ast}$ be a monoid of strings over $X$ with
identity $\varepsilon$. The {\em language } $L(\A)$ over $X$
assigned to a non-deterministic automaton is the set of elements
of $X^{\ast}$ obtainable from the labels of all possible paths of
arrows with source in $S_0$ and target in $F_0$; in this case we
say that $L= L(\A)$ is a language {\em accepted} by $\A$. We say
$w$ is {\em readable } in $\A$ from $s_1$ to $s_2$, where $s_1,
s_2 \in S(\A)$, if there is a path of arrows from $s_1$ to $s_2$
with the label $w$.

A {\em deterministic finite state automaton} can be considered a
special case of a non-deterministic finite state automaton, for
which there following conditions are satisfied: there are no
arrows labelled $\varepsilon$; each state is the source of exactly
one arrow with any given label from $X$; and $S_0$ has exactly one
element.

By Kleene-Rabin-Scott theorem the regular language over $X$ may be
identified with the language accepted by some non-deterministic
finite state automaton, or, equivalently, by some deterministic
finite state automaton. If $R = L(\A)$ is a subset of a group $F$,
then $X$ is supposed to be a set of group generators for $F$, i.e.
elements of $R$ are meant to be reduced strings over $X \cup
X^{-1}$.



\subsection{Graphs associated with a subgroup and Schreier transversals}\label{subs_graphs} We shall identify sometimes an automaton $\A$ with a finite
connected oriented labelled graph $\Gamma$ with distinguished
vertices in natural way. Namely, take $V=V(\Gamma)=S(\A)$ as its
vertex set; edge set $E=E(\Gamma)$ corresponds to arrows, with
induced labelling, and form subsets $S_0, F_0 \subseteq S(\A)$
from distinguished vertices. We shall also ascribe properties of
automata to graphs below.

 One can associate with a subgroup $C$
two graphs: \emph{the subgroup graph} $\Gamma =\Gamma_C$ and the
Schreier graph $\Gamma^\ast = \Gamma^\ast_C$; we refer reader to
\cite{km} for more details on this subject.

Recall, that $\Gamma$  is a finite connected digraph with edges
labelled by elements from $X$ and a distinguished vertex
(based-point) $1$ (so $S_0 = F_0 = \{ 1 \}$),  satisfying the
following two conditions. Firstly,  $\Gamma$ is folded, i.e.,
there are no two edges in $\Gamma$ with the same label and having
the same initial or terminal vertices. Secondly, $\Gamma$ accepts
precisely the reduced words in $X \cup X^{-1}$  that belong to
$C$. To explain the latter observe, that walking along a path $p$
in $\Gamma$ one can read a word $\ell(p)$ in the alphabet $X \cup
X^{-1}$, the label of $p$, (reading $x$ in passing an edge $e$
with label $x$ along the orientation of $e$, and reading $x^{-1}$
in the opposite direction). We say that $\Gamma$ accepts a word
$w$ if $w = \ell(p)$ for some closed path $p$ that starts at $1$
and has no backtracking. Clearly, $\Gamma$ can be identified with
a deterministic finite state  automata with $1$ as the unique
initial and final state.

The {\em Schreier graph} $\Gamma^\ast = \Gamma_C^\ast$ of $C$  is
a connected labelled digraph with the set $\{Cu \mid u \in F\}$ of
the right cosets of $C$ in $F$ as the vertex set, and such that
there is an edge from
 $Cu$ to $Cv$ with a label $x \in X$ if and only if $Cux = Cv$.  One can describe the Schreier graph $\Gamma^\ast$ as  obtained from $\Gamma$
by the following procedure. Let $v \in \Gamma$ and $x \in X$ such
that there is no outgoing or incoming
 edge at $v$ labelled by $x$. For every such $v$ and $x \in X$ we attach
to $v$ a new edge $e$ (correspondingly, either outgoing or
incoming)  labelled $x$ with a new terminal vertex $u$ (not in
$\Gamma$). Then we
 attach to $u$ the Cayley graph $C(F,X)$ of $F$ relative to $X$ (identifying $u$ with the vertex $1$ of $C(F,X)$), and
 then we fold the edge $e$ with the corresponding edge in $C(F,X)$ (that is labelled $x$ and is incoming to $u$).
Observe, that for every vertex $v \in \Gamma^\ast$ and every
reduced word $w$ in $X \cup X^{-1}$ there is a unique path
$\Gamma^\ast$ that starts at $v$ and has the label $w$.  By $p_w$
we denote such a path that starts at $1$, and by $v_w$ the end
vertex of $p_w$. The resulting graph $\Gamma^\ast_C $ is the
Schreier graph of $C$ in $F$.

Notice that $\Gamma = \Gamma^\ast$ if and only if the subgroup $C$
has finite index in $F$. A spanning  subtree $T$ of $\Gamma$  with
the root at the vertex $1$ is called {\em geodesic}  if for every
vertex $v \in V(\Gamma)$ the unique path in $T$ from $1$ to $v$
is a geodesic path in $\Gamma$. For a given graph $\Gamma$ one can
effectively construct a geodesic spanning subtree $T$.

From now on we fix an arbitrary spanning  subtree $T$ of $\Gamma$.
It is easy to see that the tree $T$ uniquely extends to a spanning
subtree $T^\ast$ of $\Gamma^\ast$.

Let $V(\Gamma^\ast)$ be the set of vertices of $\Gamma^\ast$.
Since, in general, $\Gamma^\ast$ is infinite, we need to extend
terminology from subsection \ref{subs_reg}. For a subset $Y
\subseteq V(\Gamma^\ast)$ and a subgraph $\Delta$ of
$\Gamma^\ast$, we define the \emph{language accepted by} $\Delta$
and $Y$ as the set $L(\Delta, Y,1)$ of the labels $\ell(p)$ of
paths $p$ in $\Delta$ that start at $1$ and end at one of the
vertices in $Y$, and have no backtracking. Notice that the  words
$\ell(p)$ are reduced since the graph $\Gamma^\ast$ is folded.
Notice, that $F = L(\Gamma^{\ast}, V(\Gamma^{\ast}), 1)$ and  $C =
L(\Gamma,\{1\},1) = L(\Gamma^{\ast}, \{1\}, 1)$.

Sometimes we will refer to a set of right representatives of $C$
as the {\em transversal} of $C$. Recall, that a  transversal $S$
of $C$ is termed {\em Schreier} if  every initial segment of a
representative from $S$  belongs to $S$.

 Further, let $S$ be a transversal of $C$.
 A representative $s\in S$ is
called \emph{internal} if the path $p_s$ ends in $\Gamma$, i.e., $v_s \in V(\Gamma)$.
 By $S_{\rm int}$ we denote the
set of all internal representatives in $S.$ It follows from
definition, that $|S_{\rm int}| = |V(\Gamma)|.$

 A representative $s \in S$ is called {\em geodesic}  if it
has minimal possible length in its coset $Cs$. The transversal $S$
is {\em geodesic }  if every $s \in S$ is geodesic. Clearly, if
$T^{\ast}$ is a geodesic subtree of $\Gamma^{\ast}$ (and hence $T$
is a geodesic subtree of $\Gamma$) then $S$ is a geodesic
transversal.



\section{Main results on double cosets}\label{section_properties}
In this section we study double cosets from different points of
view.

\subsection{Basic definitions and properties}\label{sub_basic}
Let $F=F(X)$ as above and suppose that $C = \langle h_1, \ldots,
h_r \rangle$ is a finite rank subgroup of $F$. Fix an element $f
\in F \smallsetminus C$ and consider a double coset $CfC$ for this
element. The following is equivalent:


\begin{align*}
g \in CfC &\Leftrightarrow \exists \,\, d_1, d_2 \in C\,\,\,\,\, g = d_1 f d_2\\
&\Leftrightarrow  \exists \,\, d_1, d_2 \in C\,\,\,\,\, d_1g =
fd_2.
\end{align*}

Denote by $E(g,f)$ the equation $xg = fy$ over a subgroup $C$.
Notice that due to the latter equivalence an element $g$ in $CfC$
iff the equation $E(g,f)$ is solvable over $C$, i.e. exist $c_1,
c_2 \in C$ such that $x=c_1, y=c_2$. Next lemmas describe some
properties of solutions of the uniform equations $E(f,f)$.

Assign with an element $f$ a subgroup $C_f \subseteq F$ such that
$C_f = C \cap C^f$, where $C^f$ is a set of all elements of the
form $f^{-1}Cf$. It turns out that this subgroup plays a special
role in the whole theory of double cosets and it is in fact the
most important characteristic of embedding of $C$ in $F$.

We define a subgroup $C$ to be called {\emph{ $f-$malnormal}} if
$C_f =1$. Recall that a subgroup $C$ is said to be
{\emph{malnormal}} if it is $f-$malnormal for all $f \in F
\smallsetminus C$. If, in contrary, $C_f$ is nontrivial then,
following \cite{BaMR}, we call an element $f$ \emph{ potentially
normalizing}.

\begin{lemma}\label{le:e(gg)} Let $C < F$ and $f \in F \smallsetminus C$. Let $D(f,f) \subseteq C \times C$ be the set of all
solutions of equation $E(f,f)$. In the notation above
\begin{itemize}
 \item[1.]  $D(f,f)$ consists of pairs $(c, c^{f^{-1}})$ for all $c
\in C_f$.\\
\item[2.] If $C$ is $f-$malnormal (malnormal {\it{ a fortiori}}), then $\sharp D(f,f) = 1$.\\
 \item[3.] If $g \in CfC$ then $C_g$ is isomorphic to $C_f$.
\end{itemize}
\end{lemma}
\begin{proof} To show 1. take an arbitrary $c_1 \in C_f$. By
definition $c_1 \in C^f \cap C$, therefore exists an element $c_2$
such that $c_1 = f^{-1}c_2 f$. This pair $(c_1, c_1^{f^{-1}})$ is
evidently provide a solution of $E(f,f)$ and all such elements can
be obtained analogously.

Let us show 2. We have $C_f = 1$, and suppose there are two
different solutions for $E(f,f)$, say $c_1, c_2$ and $d_1, d_2$ in $C \smallsetminus \{ 1 \}$.
Then $c_1 = c_2^{f^{-1}}$, $d_1 = d_2^{f^{-1}}$ and therefore $(d_1^{-1}c_1)^f=c_2d_2^{-1}$.
From this follows that $d_1^{-1}c_1, c_2d_2^{-1} \in C_f$ and then trivial, which in turns mean that $c_1 = d_1$ and
$c_2 = d_2$ and the set $D(f,f)$ consists of only one pair.

To prove claim 3. suppose $g = d_1 f d_2$ for some $d_1, d_2 \in
C$. By definition $C_g = C^g \cap C = C^{d_1fd_2} \cap C =
(C^{d_1})^{fd_2} \cap C \simeq C^f \cap C^{d_2^{-1}} \simeq C^f
\cap C = C_f$.\end{proof}


\begin{remark} If an element $f$ is pn then every other representative in $CfC$ is also potentially normalizing by lemma \ref{le:e(gg)}.
A double coset $CfC$ is called \emph{ essential} if $C_f \neq 1$ for some representative $f \in F$ (and so for all representatives of this coset).
\end{remark}

\begin{lemma}\label{le:essential} The number of different essential cosets is finite.
\end{lemma}
\begin{proof} The union of all essential double cosets forms so called generalized normalizer of $C$ in $F$, i.e. the set
$N^*_F(C) = \{ f \in F | f^{-1} C f \cap C \neq 1 \}$. This union
is known to be finite (see, for example, \cite{fmrI}, proposition
3.5.). \end{proof}
\begin{lemma}\label{le:e(fg)} Let $C < F$ and $f, g \in F \smallsetminus C$.
If the equation $E(g,f)$ has a solution $(c_1, c_2)$, where $c_1,
c_2 \in C$, then the set $D(g,f)$ of all solutions of this
equation consist of all pairs $(c_f c_1, c_2 c_f^{fc_2})$, where
$c_f \in C_f$. In particular, if $C$ is $f-$malnormal then the
solution is unique and every element $g \in CfC$ can be
represented as $g = c_1 f c_2$, where $c_1, c_2 \in C$, in a
unique way.
\end{lemma}
\begin{proof} The equation $E(g,f)$ is consistent by assumption and therefore exist a pair $(c_1, c_2)$ such that $c_1g = fc_2$. If $C$ is $f-$malnormal
then $C_f \simeq C_g = 1$ and $D(g,f)$ satisfies the derived
conclusion. Denote by $M$ the set of all such pairs and suppose
there exists another pair of solutions, say $(d_1, d_2)$. Then
$g^{-1} d_1^{-1}c_1 g = d_2^{-1}c_2$ and $f d_2 c_2^{-1} f^{-1} =
d_1 c_1^{-1}$, and therefore $d_2^{-1}c_2 = c_g \in C_g$ and
$d_1c_1^{-1} = c_f \in C_f$.  This implies $d_1 = c_fc_1$ and $d_2
= c_2 c_g$, but $c_g = c_f^{c_1g} = c_f^{fc_2}$ and $(d_1, d_2) =
(c_fc_1, c_2c_f^{fc_2}) \in M$. The opposite inclusion $M
\subseteq D(g,f)$ is straightforward.\end{proof}




\subsection{Nielsen set of generators profits}
In this subsection we look at the role of a Nielsen generating set
for $C$ which it plays in relations between elements of $C$ and
$f\in F\smallsetminus C$, how does it affect the cancellation in
cosets and double cosets of $C$. We also introduce here the notion
of a relative Schreier transversal of $C$ with respect to its
subgroup $D$.

There are a lot of ways to compute Nielsen set of generators for
$C$, see, for example \cite{LS} or \cite{km}. In this section we
assume
 that $C$ is presented by such a set $Y = \{ h_1,
\ldots, h_r  \}$. Let $S$ be a geodesic Schreier transversal for
$C$ in $F$ such that the Nielsen set of generators $Y$ has the
following properties with respect to $S$.

{\bf Properties of a Nielsen set of generators:}
\begin{itemize}
\item [i.] each $h \in Y \cup Y^{-1}$ can be written in the form
$$h = s_1 \mu(h) s_2^{-1},$$
where $s_1, s_2 \in S_{\rm int}$ are written as reduced $X-$words,
$\mu(h)$ is an element of $X \cup X^{-1}$ and
$$|l_X(s_1) - l_X(s_2)| \le 1.$$

\item [ii.] If $h_{i_1}, h_{i_2}$ are distinct elements of $Y$,
then the letters $\mu(h_{i_1})^{\pm 1}$ and $\mu(h_{i_2})^{\pm 1}$
do not cancel on computing the $X-$reduced form of $h_{i_1}^{\pm
1}h_{i_2}^{\pm 1}$.

\item [iii.] If $h_{i_1}, h_{i_2}, h_{i_3}$ are elements of $Y
\cup Y^{-1}$ such that $h_{i_2}\neq h_{i_1}^{-1}$ and $h_{i_2}
\neq h_{i_3}^{-1}$, then the letter $\mu(h_{i_2})$ does not cancel
on computing the $X-$reduced form of $h_{i_1}h_{i_2} h_{i_3}$.
\end{itemize}

Such a transversal always exists; and the letter $\mu(h)$ is
called the central letter of $h \in Y \cup Y^{-1}$. Set $M= M(Y) =
\left[\frac{1}{2}max\{l_X(h_1), \ldots, l_X(h_r)\}\right]+1.$

The following material (up to lemma \ref{le:5}) can be extracted
from \cite{BaMR}.

\begin{lemma}\cite[Lemma 2]{BaMR} Let $f \in F$ is pn and has minimal length in $CfC$.
If $f c_1 f^{-1} = c_2$, where $c_1, c_2 \in C, c_1\neq 1$, then at least one
of the letters of $c_1$ appears in the reduced form of $f c_1 f^{-1}$, i.e.,
in $c_2$.
\end{lemma}

\begin{lemma}\label{le:34}\cite[Lemma 3,4]{BaMR} Suppose $f$ is of minimal length in $fC$
and that $c = h_{i_1} \ldots h_{i_n}$ (where $h_{i_1}, \ldots,
h_{i_n} \in Y \cup Y^{-1}$) is $Y-$reduced.
\begin{itemize}

\item [1.] Then either $\mu(h_{i_1})$ remains in $fh_{i_1} \ldots
h_{i_n}$ and in this case the cancellation with $f$ in $fc$ is
exactly that of $f$ in $fh_{i_1}$; or the central letter
$\mu(h_{i_1})$ cancels in the product $fc$, and in this case
$h_{i_1}$ is of even length and exactly half of $h_{i_1}$ cancels
with $f$;

\item [2.] Suppose that $\mu(h_{i_j})$ cancels in the product
$fh_{i_1} \ldots h_{i_j} \ldots h_{i_n}$, but $\mu(h_{i_{j+1}})$
does not. Then
\begin{itemize}
\item [2.1.] $j \leq r$;


\item [2.2.] for any $k = 1, \ldots, j$ the length $l_X(h_{i_k})$
is even and $\mu(h_{i_k})$ cancels in product $fc$;


\item [2.3.] for any $k = 1, \ldots, j-1$ exactly the right half
$s_{2k}^{-1}$ of $h_{i_k}$ cancels completely in $h_{i_k}
h_{i_{k+1}}$;


\item [2.4.] $l_X(h_{i_1}) < \ldots < l_X(h_{i_j})$;


\item [2.5.] if the right half of $h_{i_j}$ does not cancel
completely with
 $h_{i_{j+1}}$, then $h_{i_1} \ldots h_{i_{j-1}}s_{1j}\mu(h_{i_j})$
is precisely the part of $c$ that cancels with $f$ and
$l_X(h_{i_1} \ldots h_{i_{j-1}}s_{1j}\mu(h_{i_j})) =
\fracd{1}{2}l_X(h_{i_j})$;

\item [2.6.] if the right half of $h_{i_j}$ does cancel with
 $h_{i_{j+1}}$, then $h_{i_1} \ldots h_{i_{j}}s$
is precisely the part of $c$ that cancels with $f$ for some $s \in
S_{\rm int}$ and $l_X(h_{i_1} \ldots h_{i_j}s) \le
\fracd{1}{2}l_X(h_{i_{j+1}})$.
\end{itemize}
\end{itemize}
\end{lemma}


\begin{lemma}\label{le:5}\cite[Lemma 5]{BaMR}
Suppose that the equation $E(f,f)$ satisfied for $c_1, c_2 \in C$, and $f$
is of minimal length in $CfC$. Then there exists
$f' \in CfC$ of the same length as $f$ which is a product of two pieces of
generators from $Y \cup Y^{-1}$. \end{lemma}

\begin{corollary}\label{cor:listessential} There exists an algorithm $A$,
which given a subgroup $C$ of $F$ and a Nielsen generated set  $Y
= \{ h_1, \ldots, h_r  \}$ lists all essential double cosets $CfC$
of $F$ (finite in their number). Moreover, $A$ runs in polynomial
time in $\max\{l_X(h_1), \ldots, l_X(h_r) \}$.
\end{corollary}

Using material above, we want to introduce a conception of relative Schreier transversals, which we will use to simplify
multiplication and representation of elements in double cosets.

For every pair $h_i, h_j \in Y^{\pm 1}$, $h_i \neq h_j^{-1}$,
define $a_{ij}$ to be the remainder
 of $h_i$ in the product $h_ih_j$, and let $b_{ij}$ be the residue of $h_j$ in the product $h_ih_j$.

Denote also $m_{ijk}$ the remainder of $h_j$ in the triple product
$h_ih_jh_k$, where $h_j \neq h_k^{-1}, h_j \neq h_i^{-1}$. Notice
that by definition of a Nielsen set of generators all words
$a_{ij}$, $b_{ij}$ and $m_{ijk}$ in $F(X)$ are nontrivial. Denote
by $\Sigma$ this
 new alphabet $\{a_{ij}\} \cup \{b_{ij}\} \cup \{m_{ijk}\} \cup \{h_i\}$  obtained from all such generators, their pairs and triples $h_i, h_j, h_k$.
We use also additional subdivision of $m_{ijk}$. Namely, let
$m_{ijk} = \a_{ij} \circ \mu_j \circ \b_{jk}$ for some $\a_{ij},
\b_{jk} \in F(X)$ (not necessarily non-trivial); sometimes we also
have needs in further partition of $\a$'-s and $\b$'s. Let then
$m_{ijk} = \a_{1ij} \circ \a_{2ij} \circ \mu_j \circ \b_{2jk}
\circ \b_{1jk}$ and we will use this notation in Section
\ref{subs_weakred}. A nontrivial $\Sigma-$reduced word $u$ is
called {\emph{ $C$-admissible}} if it has one of the following
forms
\begin{equation}\label{hihj1} c = a_{i_1 i_2} \circ m_{i_1 i_2 i_3} \circ \ldots \circ m_{i_{k-2} i_{k-1} i_k} \circ b_{i_{k-1}i_k} {\textrm{ or }}\end{equation}
\begin{equation}\label{hihj2}c = a_{i_1 i_2} \circ b_{i_1 i_2}  {\textrm{ or }}\\ c = h_i.
\end{equation}

Clearly, there is one-to-one correspondence between $C$-admissible
words and nontrivial $X-$reduced words $c \in C$.

Further, let $D$ be a nontrivial subgroup of $C$ and suppose $D$
is given by a finite Nielsen set of generators $Z = \{ d_1 ,
\ldots, d_m \}$, where $d_1, \ldots, d_m$ are $Y-$reduced words.
Consider a subgroup graph $\Gamma_D$, and take a maximal subtree
$\Upsilon$ of $\Gamma_D$ such that the corresponding Schreier
transversal $T$ respect the choice of central letters $\mu(d_i)$.
Denote by $T_{{\rm int}}$, as above, the set of all inner
representatives from $T$. Combining properties (i), (ii), (iii) of
a Nielsen set of generators with formulae (\ref{hihj1}) -
(\ref{hihj2}) above, obtain
\begin{align*}\label{relativedi} d_i &= t_{1} \cdot \mu(d_i) \cdot t_{2}^{-1}\\
&= (a_{i_1 i_2} \circ m_{i_1 i_2 i_3} \circ \ldots \circ
m_{i_{k-2} i_{k-1} i_k}) \circ m_{i_{k-1} i_{k} i_{k+1}} \circ
(m_{i_{k} i_{k+1} i_{k+2}} \circ b_{i_{k+1}i_{k+2}})  {\textrm{ or }}\\
d_i &= a_{i_1 i_2} \circ m_{i_1 i_2 i_3} \circ b_{i_2 i_3}  {\textrm{ or }}\\
d_i &= a_{i_1 i_2} \circ b_{i_1 i_2} {\textrm{  or    }} d_i =
h_{i_1},
\end{align*}

where $t_{1}, t_{2} \in T_{{\rm int}}; \,\, \mu(d_i), h_{i_1} \in
Y \cup Y^{-1}$ and $a_{ij}, b_{ij}, m_{ijk} \in \Sigma$.

The system $T_D$ of all $C-$admissible words $t \in T$ is called a
\emph{relative Schreier transversal} of $C$ in $F$ with respect to
$D$ and $\Upsilon$.


{\bf Example.} Consider an example of a relative Schreier
transversal for a subgroup $C < F(a,b)$. Let $Y = \{ a^3 , b^3 , a
b^2 a^{-1} , b a^3 b^{-1} , b a b^2 a^{-1} b^{-1} \} = \{ h_1 ,
h_2 , h_3 , h_4 , h_5 \}$ be a Nielsen set of generators of $C$.
The subgroup graph $\Gamma_C$ is shown on figure \ref{fig_subgrC},
the edges entering maximal subtree are highlighted.


\begin{figure}
\begin{center}
\psfrag{h1}{$h_1$}

\psfrag{h2}{$h_2$}

\psfrag{h3}{$h_3$}

\psfrag{h4}{$h_4$}

\psfrag{h5}{$h_5$}

\psfrag{d1}{$d_1$}

\psfrag{d2}{$d_2$}

\psfrag{d3}{$d_3$}

%
%\subfigure[Stallings automaton $\Gamma_C$ recognizing $C = \langle
%a^3 , b^3 , a b^2 a^{-1} , b a^3 b^{-1} , b a b^2 a^{-1} b^{-1}
%\rangle$.]{
%\includegraphics[scale=.6]{example.eps}
%\label{fig_subgrC} } \hspace{15mm}


\includegraphics[scale=.5]{example.eps}


\end{center}
\caption{Stallings automaton $\Gamma_C$ recognizing $C = \langle
a^3 , b^3 , a b^2 a^{-1} , b a^3 b^{-1} , b a b^2 a^{-1} b^{-1}
\rangle$}\label{fig_subgrC}
\end{figure}

Further, represent the generators from $Y \cup Y^{-1}$ in the form
$h_i = s_{1i} \circ \mu_i \circ s_{2i}^{-1}$ (uniquely determined
by $S$):

$$h_1 = a \circ a \circ a = h_6^{-1} , \,\,\,\, \,\,\,\,\, h_2 = b \circ b \circ b = h_7^{-1},$$

$$h_3 = a b  \circ b \circ a^{-1} = h_8^{-1}, \,\,\,\,\,\,\,\,\,\, h_4 = ba  \circ a \circ ab = h_9^{-1},$$

$$h_5 = b a b  \circ  b \circ a^{-1} b^{-1} = h_{10}^{-1}.$$

To construct a new alphabet $\Sigma$, one can take all suitable
products of $h_i, h_j$ and $h_k$ for $i, j, k =1, \ldots, 10$. For
instance, $a_{11} = a^3$, $a_{74} = b^{-2}$, $m_{123} = b^3$,
$m_{742} = a^3$, $b_{42} = b^2$ etc.



\begin{figure}
\begin{center}
\psfrag{h1}{$h_1$}

\psfrag{h2}{$h_2$}

\psfrag{h3}{$h_3$}

\psfrag{h4}{$h_4$}

\psfrag{h5}{$h_5$}

\psfrag{d1}{$d_1$}

\psfrag{d2}{$d_2$}

\psfrag{d3}{$d_3$}

\psfrag{a3}{$a^3$}

\psfrag{b2}{$b^2$}

\psfrag{b4}{$b^4$}



\subfigure[Subgroup graph $\Gamma_D$]{
\includegraphics[scale=.5]{example1.eps}\label{fig_subgrd} } \hspace{10mm}\subfigure[Consolidated graph $\Gamma^{\prime}_D$]{
\includegraphics[scale=.5]{consolid.eps} \label{fig_consd} } \hspace{10mm}
\end{center}
\caption{Subgroup graph $\Gamma_D$ and consolidated graph
$\Gamma^{\prime}_D$ for $D = \langle h_2^{-1} h_4 h_2, h_1 ,h_2^2
\rangle$}\label{fig_d}
\end{figure}

The element $a$ is pn and a subgroup $C_a$ is generated by a
Nielsen generating set $Z = \{ h_2^{-1} h_4 h_2, h_1 ,h_2^2 \} =
\{ d_1 , d_2 , d_3\}$. Then, taking a maximal subtree as shown on
figure \ref{fig_subgrd}, obtain $$d_1 = a_{74} \circ m_{742} \circ
b_{42}, \,\,\,\,\,\,  d_2 = h_1 \,\,\,\,\,\, {\textrm{ and }}
\,\,\,\,\,\,d_3 = a_{22} \circ b_{22}.$$

Therefore, $m_{742}, h_1$ and $a_{22}$ are central words and $1,
a_{74}, b_{42}^{-1}, b_{22}$ forms an inner part of a relative
Schreier transversal $T_{\rm int}$ for $C$ in $F$ with respect to
$C_a$. This relative transversal in terms of $X$ can be viewed on
consolidated subgroup graph $\Gamma^{\prime}_D$ on figure
\ref{fig_consd} (and we refer reader to \cite{CPAI} for more
details on consolidated graphs).

In lemma \ref{le:e(fg)} we showed that every representative $g \in
CfC$ has a unique presentation as a word $g = c_1 f c_2 $, $c_1,
c_2 \in C$ when $C_f = 1$. However, it is not the case for
essential cosets and such a presentation is not unique in general.
%%counter-example is needed
However, one can write the representatives of essential cosets in
the unique form using the following lemma.

\begin{lemma}\label{le:ess!} Let $CfC$ be an essential coset and $T$ is a
Schreier transversal for $C$ in $F$ with respect to $C_f$. Then
every $g \in CfC$ has a unique presentation
$$g = c f t, \,\,\,c_1 \in C, t \in T.$$
\end{lemma}

\begin{proof} Let $g = c_1 f d_1$ be a representative of a some essential double coset $CfC$, $c_1, d_1 \in C$. By definition of a relative Schreier
transversal there is an element $t_1 \in T$ such that $g = c_1 f
t_1$. Now suppose that $g= c_2 f t_2$ is another presentation for
$g$, $c_2 \in C, t_2 \in T$. Then $c_1ft_1 = c_2 f t_2$ implies
$c_1^{-1}c_2 = f t_1 t_2 f^{-1}$ and since $t_1 t_2 \in C$ we have
$c_1^{-1}c_2 \in C_f$ and $t_1 = C_f t_2$. But $t_1, t_2$ are both
representatives of $C_f$ in $C$ and since they are in the same
coset, $t_1 = t_2$ and hence $c_1 = c_2$. \end{proof}


\subsection{Cancellations in complex $CfC$}\label{subs_weakred}
In this subsection we estimate the size of cancellations in a
double coset depending on a subgroup $C_f$.

The following technical lemma turns out to be a crucial argument
in the proof of theorem \ref{th:bounded}.

Let $Y = \{ h_1, \ldots, h_r  \}$ be a fixed set of Nielsen
generators for $C$. Remind that $M=\left[\frac{1}{2}max\{l_X(h_1),
\ldots, l_X(h_r) \}\right]+1$ and set $p$ being equal to the
number of elements in the ball of radius $2M$ in $F(X)$. We will
use this notation in subsequent.

\begin{lemma}\label{le:key} Let $f_0, f_1 = h_{i_1} f_0 h_{j_1},
\ldots, f_k = h_{i_k} f_{k-1} h_{j_k}$ be a sequence of elements
in $F(X)$ such that $h_{i_1}, h_{j_1}, \ldots, h_{i_k}, h_{j_k}$
are elements of $Y \cup Y^{-1}$; $h_{i_k} \dots h_{i_1}$ and
$h_{j_1} \dots h_{j_k}$  are $Y-$reduced words, and $l_X(f_l) \leq
2M$ for all $l = 1, \ldots, k$. Then for all $k \geq p$ there are
$l,n \in \{ 1, \ldots, k \}$ and nontrivial $c_l, c_n, d_l, d_n
\in C$ such that $ l \neq n$, and $c_l\neq c_n, d_l \neq d_n$, and
$$c_lf_ld_l= c_nf_ld_n = c_nf_nd_n= c_lf_nd_l.$$
\end{lemma}

\begin{proof} Fix a number $k$; notice that $M$ is at least $1$ so $p \geq 4
m^2 +1$ for a free group $F$ of rank $m$ and therefore $k \geq p
\geq 17.$ %��� ����� ����� $k$ ���� �� ��������� ($ = 1,2$), � ������������
 In particular, $k > 0$. The number of all elements of length not greater then $2M$ is equal to $p$ by definition, and so by Dirichlet's drawer
principle there are at least two equal elements among $k+1>p$
elements of such length, say $f_l$ and $f_n$, and $l < n$. The
statement of lemma is obvious for $f_l = f_n =1$, so suppose $f_l
, f_n$ are non-trivial.

Since

$$ h_{i_k} \dots h_{i_1} f_0 h_{j_1} \dots h_{j_k} = h_{i_k} \dots h_{i_{l+1}} f_{l} h_{j_{l+1}} \dots h_{j_k}
= f_k, $$ one can take $c_l =h_{i_k} \dots h_{i_{l}}$, $c_n =
h_{i_k} \dots h_{i_{n}}$ and $d_l =h_{j_l} \dots h_{i_{k}}$, $d_n
=h_{j_n} \dots h_{i_{k}}$. Clearly, $c_l,c_n, d_l,d_n$ are
non-trivial (as $Y-$reduced product of non-trivial elements ) and
$c_l \neq c_n$,  $d_l \neq d_n$. Indeed, let, for instance, $c_l
=c_n$, then the product $c_l^{-1} c_n$ is equal to $h_{i{l+1}}
\dots h_{i_k}$. But $h_{i{l+1}} \dots h_{i_k}$ can not be trivial
as $Y-$reduced word in Nielsen generators. Therefore, $c_lf_ld_l=
c_nf_ld_n$ and other equalities are straightforward. \end{proof}

Let $A_1, A_2, \ldots, A_n$ be a finite number of subsets in a
free group $F(X)$ and consider a function $cn: A_1 \cdot A_2 \cdot
\ldots \cdot A_n \mapsto \mathbb{N}$ which computes the total
amount of $X-$cancellations in a product $a_1 \cdot \ldots \cdot
a_n$:

$$cn(a_1, a_2, \ldots, a_n)
=\fracd{1}{2}\left(\mathop{\sum}\limits^{n}_{i=1} l_X(a_i) -
l_X(a_1 a_2 \cdot \ldots \cdot a_n)\right).$$


Observe that for $n=2$ the function $cn(a_1, a_2)$ coincide with
the notion of Lyndon-Chiswell-Gromov product $(a_1, a_2^{-1})$ in
a group $F$ with respect to canonical length function $l_X$ (see
\cite{LY,Chi} for details on length functions, and in \cite{Gr}
this definition was adapted to hyperbolic metric spaces).
 We say that the product $A_1
\cdot A_2 \cdot \ldots \cdot A_n$ is \emph{$k-$reduced} if there
is a constant $k \ge 0$ such that for every $a_i \in A_i, i=1,
\ldots , n$ holds $$cn(a_1, a_2, \ldots, a_n) \leq k.$$
%%so the depth of cancellations is $k$
For instance, two singletons $A_1 = \{ u\}, A_2 = \{v \}$ for
which $uv = u\circ v$ forms a $0-$reduced set, but we will omit
the prefix $0-$ below saving ``$k-$'' only for $k >0$.



\begin{theorem}\label{th:bounded} Let $F$ be a free group with a
finite bases $X$ and $C$ be a finitely generated subgroup of $F$
with a Nielsen set of generators $Y$. Suppose $f \in F
\smallsetminus C$ and $f$ is of minimal length in a double coset
$CfC$, and let $k = 2pM$.

\begin{itemize} \item[1.] If $C$ is $f-$malnormal, then
$CfC$ is $k-$reduced.

\item[2.] If $CfC$ is essential, then $CfT$ is $k-$reduced, where
$T$ is a relative Schreier transversal of $C$ in $F$ relative to
$C_f$.
\end{itemize}
 \end{theorem}


\begin{proof} Let $g \in CfC$. If the element $f$ does not cancel completely
in $g = c f d$ then by lemma \ref{le:34} the length of
cancellations $cn(c,f) \le M$ (as well as $cn(f,d)\le M$). Hence
$cn(c,f,d) = cn(c,f)+cn(f,d) \leq 2M$ and both (1.) and (2.)
follows.
%%$= c f d$
%%for some $c, d \in C$.

If, in contrary, $f$ cancels completely, consider two following
cases.

{\bf 1.} Suppose $C_f= 1$ and $g \in CfC$ such that the
cancellations in $cfd = g$ are greater then $k$ for $c = h_{i_1}
\ldots h_{i_k}, d = h_{j_1} \ldots h_{j_n} \in C$.

Let $f = f' \circ f''$, where $f'$ cancels completely in $cf$ and
$f''$ cancels in $fd$. Below we list all possible forms of pieces
$f'$ and $f''$.

By lemma \ref{le:34} $f''$  possess the following properties:



\begin{itemize}
\item [1.1.] Either $(f'')^{-1} = s$, where $s \in S_{\rm int}$
and $s$ is initial segment of $s_{1j_1}$. Then

$f'' d = \g h_{j_2} \cdots h_{j_n}$ and $\g^{-1}$ is a beginning
of $h_{j_1}^{-1}$ of length $l_X(\g) < l_X(h_{j_1}) \le 2M$, and $l_X(f'') \le M$;  or\\

\item [1.2.] $(f'')^{-1} = s_{1j_1} \mu_1$. Then

$f'' d = s_{2j_1}^{-1} h_{j_2} \cdots h_{j_n}$ and $l_X(s_{2j_1}^{-1}) \le M$, $l_X(f'') \le M$;  or \\


\item [1.3.]$(f'')^{-1} = a_{j_1 j_2} \circ m_{j_1 j_2 j_3} \circ
m_{j_2 j_3 j_4} \circ \cdots \circ \circ m_{j_{s-2} j_{s-1} j_s}
\circ \a \circ \mu_{j_s}$, in which case


 $f'' d = \b h_{j_{s+1}} \cdots
h_{j_n}$; recall that $\a,\b$ defined by $m_{j_{s-1} j_{s}
j_{s+1}}=\a \circ \mu_{j_s} \circ \b$. Here the length $l_X(f'') \le M$ (see lemma \ref{le:34}, (2.5)) and $l_X(\b) \le M$;  or \\

\item [1.4.] $(f'')^{-1} = a_{j_1 j_2} \circ m_{j_1 j_2 j_3} \circ
m_{j_2 j_3 j_4} \circ \cdots \circ \circ m_{j_{s-2} j_{s-1} j_s}
\circ \a_1 $, where

 $f'' d = \a_2 \mu_{j_s} \b h_{j_{s+1}} \cdots
h_{j_n}$; and $\a_1, \a_2, \b$ are such that $m_{j_{s-1} j_{s}
j_{s+1}}=\a_1 \a_2 \circ \mu_{j_s} \circ \b$. Here the length
$l_X(f'') \le M$ (see (2.6) of lemma \ref{le:34}) and
$l_X(\a_2 \mu_{j_s} \b) \le l_X(h_{j_s}) \le 2 M$.\\


\end{itemize}
Similarly, $f'$ and $cf'$ have one of the forms:

\begin{itemize}
\item [2.1.] Either $(f')^{-1} = s$, where $s \in S_{\rm int}$ and
$s$ is terminal segment of $s_{2i_k}$. Then

$c f' = h_{i_1} \cdots h_{i_{k-1}}  \g $ and $\g^{-1}$ is a
beginning
of $h_{i_k}$ of length $l_X(\g) \le 2M$, and $l_X(f') \le M$;  or\\

\item [2.2] $(f')^{-1} =  \mu_1 s_{2i_k}^{-1}$. Then

$f' c =  h_{i_1} \cdots h_{i_{k-1}} s_{1i_k} $ and $l_X( s_{1i_k}) \le M$, $l_X(f') \le M$;  or \\


\item [2.3.]$(f')^{-1} = \mu_{i_l}  \circ \b \circ m_{i_l i_{l+1}
i_{l+2}} \circ m_{i_{l+1} i_{l+2} i_{l+3}} \circ \cdots \circ
m_{i_{k-2} i_{k-1} i_k} \circ b_{i_{k-1} i_k} $, in which case


 $c f' =  h_{i_1} \cdots
h_{i_{l-1}} \a$; here $\a,\b$ defined by $m_{i_{l-1} i_{l}
i_{l+1}}=\a \circ \mu_{i_l} \circ \b$. Here the length $l_X(f') \le M$ and $l_X(\a) \le M$;  or \\

\item [2.4.] $(f')^{-1} = \b_2 m_{i_l i_{l+1} i_{l+2}} \circ
m_{i_{l+1} i_{l+2} i_{l+3}} \circ \cdots \circ m_{i_{k-2} i_{k-1}
i_k} \circ b_{i_{k-1} i_k}  $, where

 $c f' =  h_{i_1} \cdots
h_{i_{l-1}}   \a  \circ  \mu_{i_l}  \circ \b_1$; and $\b_1, \b_2,
\a$ are such that $m_{i_{l-1} i_{l}
i_{l+1}}=\a \circ \mu_{i_l} \circ \b_1 \b_2$. Here the length $l_X(f') \le M$ and $l_X(\a \mu_{i_l} \b_1) \le 2M$.\\
\end{itemize}


Summarizing conditions (1.1) -- (2.4) and renumbering for
notational simplicity $h_{i_1}, \ldots, h_{j_n}$ if necessary,
obtain

\begin{equation}\label{q1q2}
g= h_{i_1} \cdots h_{i_k} f' f''  h_{j_1} \cdots h_{j_n} = h_{i_1}
\cdots h_{i_{l-1}} q_1 q_2 h_{j_{s+1}} \cdots h_{j_n},
\end{equation}
where
$$
q_1 = \left\{
\begin{aligned}
\g_1, {\textrm{ see (2.1), and $l_X(\g_1) \le 2M$,}} \\
s_{1i}, {\textrm{ see (2.2-2.3), and $l_X(s_{1i}) \le M$,}} \\
s_{1i} \mu_i \b_1, {\textrm{ see (2.4), and $l_X(s_{1i} \mu_i \b_1) \le 2 M$}}, \\
\end{aligned}
\right.
$$
and

$$
q_1 = \left\{
\begin{aligned}
\g_2, {\textrm{ see (1.1), and $l_X(\g_2) \le 2M$,}} \\
s_{2j}^{-1}, {\textrm{ see (1.2-1.3), and $l_X(s_{2j}) \le M$,}} \\
\a_2 \mu_i s_{2j}^{-1} , {\textrm{ see (1.4), and $l_X(\a_2 \mu_i s_{2j}^{-1}) \le 2 M$}}. \\
\end{aligned}
\right.
$$

If $q_1 q_2 = q_1 \circ q_2$, then again $cn(c,f,d) = cn(c,f) +
cn(f,d)$ and the total cancellations in $cfd$ are bounded again by
$2M$, contradiction with assumption $cn(c,f,d) > k$. Therefore,
$q_1 q_2$ is not reduced. Notice, that $f \notin C$ implies $g
\notin C$ and therefore $q_1 q_2 \notin C$.  Suppose, $q_1 q_2$
cancels in $q_1' \circ q_2'$, where $q_1'$ is an initial part of
$q_1$, $q_2'$ is a terminal part of $q_2$. Without loss of
generality one can assume that all $h_{i_1}, \ldots, h_{j_n}$
cancels in product

$$c f d = h_{i_1} \cdots h_{i_{l-1}} q_1' \cdot q_2' h_{j_{s+1}} \cdots h_{j_n}.$$

By definition of cancellation function we have

\begin{equation}\label{cn(cfd)}
cn(c,f,d) = \fracd{1}{2}\left( l_X(c) +l_X(f)+l_X(d) -
l_X(g)\right),
\end{equation}

where $$l_X(c) = \mathop{\sum}\limits_{t=1}^{k}l_X(h_{i_t}) - 2
\mathop{\sum}\limits_{t=1}^{k-1}cn(h_{i_t}, h_{i_{t+1}}),
$$
$$l_X(d) = \mathop{\sum}\limits_{z=1}^{n}l_X(h_{j_z}) - 2
\mathop{\sum}\limits_{z=1}^{n-1}cn(h_{j_z}, h_{j_{z+1}}),
$$
and $l_X(f) = l_X(f')+ l_X(f'')$.

Rearrangement of summands in (\ref{cn(cfd)}) and (\ref{q1q2}) give
us
$$ cn(c,f,d) = cn(h_{i_{l}} \cdot \ldots \cdot h_{i_k}, f')+  cn( f'', h_{j_{1}} \cdot \ldots \cdot
h_{j_{s}})-$$
$$- \mathop{\sum}\limits_{t=1}^{l-1}cn(h_{i_t},
h_{i_{t+1}}) - \mathop{\sum}\limits_{z=s}^{n-1}cn(h_{j_z},
h_{j_{z+1}})+ $$

$$+ \fracd{1}{2} \left( l_X(q_2)+ l_X(q_1)+
\mathop{\sum}\limits_{t=1}^{l-1}l_X(h_{i_t})+
\mathop{\sum}\limits_{z=s+1}^{n}l_X(h_{j_z}) - l_X(g)\right).$$



Further, formulae (1.1) - (2.4) implies $cn(h_{i_{l}} \cdot \ldots
\cdot h_{i_k}, f') \leq M$, $cn( f'', h_{j_{1}} \cdot \ldots \cdot
h_{j_{s}}) \leq M$, and $l_X(q_1) \leq 2M, l_X(q_2) \leq 2M$. By
definition of Nielsen set of generators $cn(h_i, h_j) \geq 0$ and
$l_x(h_i) \leq 2 M$ for all $h_i, h_j \in Y \cup Y^{-1}$.
Therefore,

$$k < cn(c,f,d) \leq (l+n-s+3) M - \fracd{1}{2} l_X(g).$$

Moreover, the assumption that all $h_{i_1}, \ldots, h_{j_n}$
cancels in (\ref{q1q2}) implies $l-1 = n-s$ and

\begin{equation}\label{g=ab}
g = a_{i_1i_2}' \circ b_{j_1 j_2}',
\end{equation}

where $a_{i_1i_2}'$ is an initial segment of $a_{i_1i_2} \in
\Sigma$, $b_{j_1 j_2}'$ is a terminal segment of $b_{j_{n-1} j_n}
\in \Sigma$ and $g$ is not $C-$admissible. Hence, $(i_1, i_2) \neq
(j_{n-1}, j_n)$ and since $l_X(g)
> 0$ we have
\begin{equation}\label{lp}
0 < l_X(g) \leq 4M (l+1) - 4 p M \,\,\,\,\Rightarrow \,\,\,\, l >
p-1.
\end{equation}

Consider the sequence

\begin{align*}
f_0 &= f,\\
f_1 &= h_{i_{l}} \cdot \ldots \cdot h_{i_k} f_0 h_{j_{1}} \cdot
\ldots \cdot
h_{j_{s}},\\
f_2 &= h_{i_{l-1}} f_1 h_{j_{s+1}},\\
&\ldots,\\
f_{l-1} &= h_{i_2} f_{l-2} h_{j_{n-s}},\\
f_{l} &= g.
\end{align*}
Here $l_X(f_0) = l_X(f') + l_X(f'')$ and by (1.1) - (2.4) we have
$l_X(f_0) \le 2M$ and by formula (\ref{g=ab}) one can easily
deduce $l_X(f_{l}) \le 2M$.

Further, since $f_1 = q_1' \circ q_2'$, and $c, d$ are $Y-$reduced
words in a Nielsen set of generators $Y$ (see properties (i) -
(iii)), the equality (\ref{g=ab}) is possible only if all lengths
$l_X(f_1), \ldots, l_X(f_{l-1})$ are also bounded above by $2M$.

Observe that $l > p -1$ due to formula (\ref{lp}), and $f_0,
\ldots, f_{l}$ represent the same double coset. Hence by lemma
\ref{le:e(gg)} (3.) we have $C_f = C_{f_0} \simeq \ldots \simeq
C_{f_{l}} \simeq
 C_{f_{l+1}}= C_g$.

Therefore, $f_0, \ldots, f_{l}$ satisfy the assumptions of lemma
\ref{le:key} and hence there are elements, say $f_i, f_j, c_i,
d_i, c_j$ and $d_j$
 such that $c_i \neq c_j, d_i \neq d_j$ but $c_i f_i d_i = c_j f_i
 d_j$, contradiction with uniqueness of representative $f_i$ in $CfC$ (see lemma \ref{le:e(fg)}).


{\bf 2.} Let $C_f \neq 1$. Then by lemma \ref{le:ess!} there is a
unique presentation $g = c f d$ for $c \in C$ and $d \in T$ for a
relative Schreier transversal $T=T_{C_f}$. Then considering
elements $d \in T$ and arguing as above, obtain for different
indexes $i,j$ the equality $g = c_i f_i d_i = c_j f_i d_j$ again,
which now contradicts with lemma \ref{le:ess!}.\end{proof}

\begin{remark} Notice that for essential double cosets the complex $CfC$
might not be $k-$reduced for any $k$. Indeed, let $w$ be a
primitive element of an arbitrary finite rank free group $F$, and
let $C$ be the subgroup of $F$ generated by $w^l$ for some $l>1$.
Then $f=w$ is pn and does not belong to $C$. However, the total
amount of cancellations in a product $w^{kl} w w^{-kl} \in CfC$
can be arbitrary large.
\end{remark}


\subsection{Formal language and automatic properties}\label{section_measurable}

In this subsection we investigate connections between bounded
cancellations and regularity of subsets of $F(X)$, and using these
relations we establish the regularity of all reduced words
representing elements of a double coset $CfC$ in a free group $F$.

\begin{theorem}\label{th:weakred} Suppose $A_1, A_2$ are two regular subsets in $F$. If
the product $A_1A_2$ is $k-$reduced, then the set
$\overline{A_1A_2}$ of all reduced words in $A_1A_2$ is regular in
$F(X)$.
\end{theorem}
\begin{proof} Let $\A_i=(S(\A_i),X,\d_i,S_0(\A_i),F_0(\A_i))$
be deterministic automata accepting $A_i$; in particular, it means
$S_0(\A_i) = \{s_0(\A_i)\}$ , $i = 1, 2$. One can easily form a
(non-deterministic) automaton $\A$, which accepts the
concatenation $A_1A_2$. Namely, take the same alphabet $X$, $S(\A)
= S(\A_1) \cup S(\A_2)$, the new start state is $s_0(\A_1)$, the
final states will be $F_0(\A) = F_0(\A_1) \cup F_0(\A_2)$, with
the union of all labels and additional $\varepsilon-$transitions
from all states $F_0(\A_1)$ to $s_0(\A_2)$. Since $A_1A_2$ is
$k-$reduced, the length of cancellations between the elements $a_1
a_2$, where $a_1 \in A_1, a_2 \in A_2$ is bounded by $k$.
Therefore, a set $U$ of words $u_j \in F(X)$ such that $a_1 = b_1
\circ u_j^{-1}$, $a_2 = u_j \circ b_2$, and $a_1 a_2 = b_1 \circ
b_2$, where $a_i \in A_i$, $b_i \in F(X)$, is finite. Since $\A_1$
is a finite state automaton, for every $u_j \in U$ there is a
finite set $P_{u_j} = \{ p_{u_j} \in S(\A_1): {\textrm{ $u^{-1}$
is readable from }} p_{u_j} {\textrm{ to $f_0$ for some $f_0 \in
F_0(\A_1)$}}\}$; by the same argument the set $Q_{u_j} = \{
q_{u_j} \in S(\A_2): {\textrm{ $u$ is readable from }} s_0(\A_2)
{\textrm{ to $q_{u_j}$ }} \}$ is finite. For all $u_j$ add
$\varepsilon-$transitions from the states of $P_{u_j}$ to the
states of $Q_{u_j}$; these complete the definition of $\A$.
Clearly, $\overline{A_1A_2} \subseteq L(\A)$ by construction, and
moreover, $\overline{A_1A_2} = L(\A) \cap F(X)$. Thus,
$\overline{A_1A_2}$ is regular as the intersection of regular
sets.
\end{proof}

A {\em double-based cone } with bases $w_1, w_2$ is a set of all
reduced words in $F(X)$, starting with $w_1$ and ending with
$w_2$.
\begin{corollary}\label{cor:cones_reg} Every double-based cone $C(w_1,w_2)$ in a finite rank free group is a regular set.
\end{corollary}
\begin{proof} One can prove this statement using theorem
\ref{th:weakred}, but we shall give here its direct proof. Let
$w_1 = x_{i_1} \cdots x_{i_s}$, $w_2 = x_{j_1} \cdots x_{j_t}$.
Then $f \in C(w_1,w_2)$ by definition have a form $f = x_{i_1}
\cdots x_{i_s} \circ y \cdots z \circ x_{j_1} \cdots x_{j_t}$,
thus $y \neq x_{i_s}^{-1}$, and $z \neq x_{j_1}^{-1}$, from which
follows that it is sufficient to prove the regularity of a cone
with bases $x_{i_s}$ and $x_{j_1}$. Indeed, if $C(x_{i_s},
x_{j_1})$ is regular, then so is $C(w_1, w_2) = x_{i_1} \cdots
x_{i_{s-1}} \circ C(x_{i_s}, x_{j_1}) \circ x_{j_2} \cdots
x_{j_t}$ as it is a concatenation of regular sets in $F(X)$. For
the notational simplicity we construct a deterministic automaton
$\A$ recognizing a cone $C(x_{i_s}, x_{j_1})$ in $F(X) = F(a,b)$
and for bases equal to $a$ and $b$ respectively. An automaton $\A$
recognizing $C(a,b)$ is shown on figure \ref{fig_Cab} (a tailed
arrow corresponds to the initial state with an arrow, and final
states are labelled by double circles). If now the rank of a free
group $F$ is greater than $2$, $\A$ can be easily modified to
recognize $C(a,b)$ in $F=F(X)$. Namely, it is necessary to add the
states corresponding to the rest of generators $X \cup X^{-1}$.
From every state $q_{x_i}$ we add transitions from this state to
all other including this one and excluding $q_{x_i^{-1}}$, the
state corresponding to inverse of generator; from all other states
excluding $q_{x_i^{-1}}$ we add transitions to $q_{x_i}$, and the
result follows.
\begin{figure}
\begin{center}
\includegraphics[width=6cm]{figure1}\\
\caption{The automaton $\A$ recognizing the cone
$C(a,b)$.}\label{fig_Cab}
\end{center}
\end{figure} \end{proof}


\begin{theorem}\label{th:cfcreg} Let $C$ be a finite rank subgroup of a free
group $F(X)$. Then the set of all $X-$reduced words representing
elements of $CfC$ is regular in $F(X)$.
\end{theorem}

\begin{proof} If $C_f = 1$, then by lemma \ref{le:e(fg)} all $X-$reduced words can be represented by elements of $CfC$, and if
$C_f \neq 1$ then by lemma \ref{le:ess!} these words are
represented by elements of a complex $CfT$ for a relative Schreier
transversal $T$ of $C$ with respect to $C_f$. In both cases sets
$CfC$ and $CfT$ are $k-$reduced in $F(X)$ by theorem
\ref{th:bounded}. Further, since $C$, $fC$ and $fT$ are regular in
$F(X)$, the result follows
 by theorem \ref{th:weakred}. \end{proof}



%%%%%%%%%%%%%%%%%%%%%%%%%%%%%%%%%%%%%%%%%%%%

\begin{thebibliography}{\hspace{0.5in}}

\bibitem{BaMR} G.~Baumslag, A.~G.~Miasnikov, V.~N.~Remeslennikov, {\it
Malnormality is decidable in free groups}. Intern. J. of Algebra
and Computation, 9 (1999), no. 6, 687 -- 692.

\bibitem{BeFe} M.~Bestvina, M.~Feighn, {\it A combination theorem for negatively
curved groups}. Intern. J. Differential Geom. v.35, no. 1 (1992),
85 -- 101.

\bibitem{CPAI}  A.~V.~Borovik, A.~G.~Myasnikov and V.~N.~Remeslennikov,
{\it The Conjugacy Problem in Amalgamated Products I: Regular
Elements and Black Holes}. Intern. J. of Algebra and Computation,
17 (2007), no. 7,  1301 -� 1335.

\bibitem{BriWi} M.~Bridson and D.~Wise, {\ Malnormality is undecidable in hyperbolic
groups}. Israel J.of Math., 124 (2001), 313 -- 316.

\bibitem{Chi} I.~Chiswell, {\it Abstract length functions in groups}. Math. Proc. Cambridge Phil. Soc., 80 (1976), 451 -- 463.

\bibitem{eps} D.~Epstein, J.~Cannon, D.~Holt, S.~Levy, M.~Paterson and
W.~Thurston, {\it Word Processing in Groups}. Jones and Bartlett,
Boston, 1992.


\bibitem{fmrI} E.~Frenkel, A.~G.~Myasnikov and V.~N.~Remeslennikov,
{\it Regular sets and counting in free groups}. Combinatorial and
Geometric Group Theory, series ``Trends in Mathematics'', 2010
Birkhauser Verlag Basel/Switzerland, 93 -- 118.

\bibitem{fmrII}  E.~Frenkel, A.~G.~Myasnikov and V.~N.~Remeslennikov,
{\it Amalgamated products of groups: measures of random normal
forms}. Fundamental and Applied Mathematics, v. 16, no. 8 , 2010,
189 -- 221 (see also in http://arxiv.org/abs/1107.4079v1).

\bibitem{Gr} M.~Gromov, {\it Hyperbolic groups}. Essays in group theory, S. M.
Gersten (ed.), MSRI 8, Springer-Verlag, Berlin, 1987, 75 -- 263.



\bibitem{Jitsukawa} T.~Jitsukawa, {\it Malnormal subgroups of free
groups}. Computational and statistical group theory (Las Vegas,
NV/Hoboken, NJ, 2001), Contemp. Math. 298, Amer. Math. Soc.,
Providence, RI, 2002,  83 -- 95.

\bibitem{km} I.~Kapovich and A.~G.~Myasnikov, {\it Stallings foldings and
subgroups of free groups}. J.of Algebra 248, 2002, 608 -- 668.

\bibitem{khm} O.~Kharlampovich and A.~Myasnikov, {\it Hyperbolic groups and free
constructions}. Transactions of the A.M.S., v. 350, no. 2, 1998,
571--613.

\bibitem{LY} R.~Lyndon, {\it Length functions in groups}. Math. Scand., 12, 1963, 209 -- 234.

\bibitem{LS} R.~C.~Lyndon and P.~Schupp, {\it Combinatorial group
theory}. Ergebnisse der Mathematik und ihrer Grenzgebiete v.~89,
Springer-Verlag, Berlin, Heidelberg, New York, 1977.

\bibitem{MO} K.~V.~Mikhajlovskii and A.~YU.~Ol'shanskii, {\it Some
constructions relating to hyperbolic groups}. London Math. Soc.
Lecture Notes Ser., 252 (1998), 263 -- 290.


\end{thebibliography}

%%\bigskip
%%
%%\normalsize
%%
%%\vfill
%%
%%\noindent \textsf{Elizaveta Frenkel, Moscow State University,
%%GSP-1, Leninskie gory, 119991, Moscow, Russia}
%%
%%\noindent {\tt lizzy.frenkel@gmail.com}
%%
%%
%%\medskip
%%\noindent \textsf{Vladimir N. Remeslennikov, Omsk Branch of
%%Mathematical Institute SB RAS, 13 Pevtsova Street, Omsk 644099,
%%Russia}
%%
%%\noindent {\tt remesl@iitam.omsk.net.ru}

\end{document}
