\documentclass[a4paper,12pt]{article}
\usepackage{amsmath,amsfonts,amsthm,amscd,amssymb,latexsym,comment,eufrak}
%%%%%%%%%%%%%
\usepackage{enumerate,graphicx,psfrag}%,subfigure,jchangebar,oldgerm}
\usepackage[mathscr]{eucal}
\usepackage[usenames]{color}
%\usepackage{showkeys}
%%%%%%%%%
\sloppy
%%%%%%%%%%%%%%%%%%%%%%%
%
%
%
\title{The subgroup membership problem
}
\author{ 
}

%%%%%%%%%%%%%%%%%%%%%%%%%%%

%%%%%%%%%%%%%%%%%%%%%%%%%%%
\def\nul{\emptyset }
\def\D{\Delta }
\def\d{\delta }
%\def\vd{\vardelta}
\def\b{\beta }
\def\y{\zeta }
\def\i{\iota }
\def\k{\kappa }
\def\l{\lambda }
\def\L{\Lambda }
\def\r{\rho }
\def\x{\chi }
\def\ep{\epsilon}
\def\e{\varepsilon }
\def\G{\Gamma }
\def\g{\gamma }
\def\a{\alpha }
%\def\t{\tau }
\def\th{\theta}
\def\s{\sigma }
\def\S{\Sigma }
\def\W{\Omega}
\def\w{\omega }
\def\pd{\partial}
\def\wht{\widehat}
%\def\cC{{\mathcal C}}
%\newcommand{\cdim}{\texttt{cdim}}
\def\fC{{\textswab C}}
%
\def\cD{{\cal{D}}}
\def\cF{{\cal{F}}}
\def\cH{{\cal{H}}}
\def\cJ{{\cal{J}}}

\def\cK{{\cal{K}}}
\def\cP{{\cal{P}}}
\def\cR{{\cal{R}}}
\def\cS{{\cal{S}}}
\def\cW{{\cal{W}}}
\def\cQ{{\cal{Q}}}
%\newcommand{\GG}{\ensuremath{\mathbb{G}}}
%%%%%%%%%%%%%%%%%%%%%%%%%%%%%%


%%%%%%%%%%%%%%%%%%%%%%%%%%%%%%
\newtheorem{theorem}{Theorem}[section]
\newtheorem{lemma}[theorem]{Lemma}
\newtheorem{corollary}[theorem]{Corollary}
\newtheorem{proposition}[theorem]{Proposition}
\newtheorem{axiom}[theorem]{Axiom}
\newtheorem{definition}[theorem]{Definition}
\newtheorem*{defn*}{Definition}
\newtheorem{conjecture}[theorem]{Conjecture}
%cvs -d :pserver:najd2@cvs.mas.ncl.ac.uk:/CVS/najd2
\newtheorem{exam}[theorem]{Example}
%
%
\newenvironment{example}{\begin{exam} \rm}{\end{exam}}
%
%
%
\newtheorem{remk}[theorem]{Remark}
\newenvironment{remark}{\begin{remk} \rm}{\end{remk}}
%
%%%%%%%%%%%%
\numberwithin{equation}{section}
\numberwithin{figure}{section}
%%%%%%%%%%%%%%%%%%%%
\newcommand{\Iso}{\operatorname{Isom}}
\newcommand{\Aut}{\operatorname{Aut}}
%%%%%%%%%%%%%%%%%%%
\renewcommand{\AA}{\ensuremath{\mathbb{A}}}
\newcommand{\ZZ}{\ensuremath{\mathbb{Z}}}
\newcommand{\QQ}{\ensuremath{\mathbb{Q}}}
\newcommand{\RR}{\ensuremath{\mathbb{R}}}
\newcommand{\NN}{\ensuremath{\mathbb{N}}}
\newcommand{\CC}{\ensuremath{\mathbb{C}}}
%\renewcommand{\ker}{\verb"Ker"}
\newcommand{\cC}{\mathcal{C}}
\newcommand{\cO}{\mathcal{O}}
\newcommand{\la}{\langle}
\newcommand{\ra}{\rangle}
%\newcommand{\BA}{\ensuremath{\mathbb{A}}}
%%%%%%%%%%%%%%%%%%%%%%%%%%%%%%%%%%%%%%
\newcommand{\maps}{\rightarrow}
\newcommand{\ov}[1]{\overline{#1}}
\newcommand{\bs}{\backslash}
%%%%%%%%%%%%%%%%%%%%%%%%%%%%%%%
\newcommand{\be}{\begin{enumerate}}
\newcommand{\ee}{\end{enumerate}}
\newcommand{\bd}{\begin{description}}
\newcommand{\ed}{\end{description}}
%%%%%%%%%%%%%%%%%%%%%%%%%%%%%%%%%%% 
%
\newenvironment{ajd}{\noindent\color{blue} AJD }{}
\newenvironment{xxx}{\noindent\color{red}}{}
%
\begin{document}
\maketitle
\section{Double coset normal form}
Based on notes copied (laboriously) from Christian's page 
\begin{verbatim}
http://checkmyworking.com/misc/writemaths/?doublecoset
\end{verbatim}
$F_1$, $F_2$ free groups (finitely generated by $X_1$ 
and $X_2$, respectively).

Let $H_1 \leq F_1$, $H_2 \leq F_2$ such that 
there exists an isomorphism $\phi: H_1 \rightarrow H_2$.

i.e. we have finite bases $\{h_1, \ldots, h_m \}$  for $H_1$ and 
$\{h_1', \ldots, h_m'\}$ for $H_2$.

Consider the free group $F(Z)$, generated by $Z=\{z_1, \ldots, z_m\}$, 
and the maps $\phi_1$ and $\phi_2$ such that 
$\phi_1(z_i)=h_i(X_1)  $ and  $\phi_2(z_i)=h^\prime_i(X_2)$, $i=1,\ldots ,m$.
Thus $\phi_i$ extends to an isomorphism of $F(Z)$ to $H_i$. 

Let ${G = F_1 \underset{H_1=H_2}{\ast} F_2}$, the group with  
 presentation $\la X_1,X_2 | h_i = h_i', i=1 \ldots m\ra$.

A  set $S_i \subseteq F_i$ such that
\be 
\item
$F_i = \displaystyle{\bigcup_{s \in S_i} H_isH_i}$ 
and 
\item
for all $s, s^\prime \in S_i$, $s\in H_i s^\prime H_i$ 
implies $s=s^\prime$,
\ee
is called a set of \emph{double coset representatives for} $H_i\le F_i$. 
(We shall assume $S_i$ is infinite.) 

Let $S_1$ and $S_2$ be sets of  double coset representatives of 
$H_1\le F_1$ and $H_2\le F_2$, respectively. Let $g \in G$. 
A word $w$ representing $g$ is in \emph{(double coset) normal form} if 
$w = h_{1}p_1h_{2}p_2 \cdots h_{k}p_kh_{{k+1}}$, with
\be
\item $p_i \in S_1\cup S_2$  and $h_i \in F(Z)$ and
\item if $p_i\in S_j$ then $p_{i+1}\notin S_j$, for $j=1$ and $2$,
\ee
for $i = 1 \ldots k+1$.
\section{Finding double coset representatives}
\begin{comment}
Suppose $g = g_1g_2 \cdots g_k$, where $g_i$ is in a factor $F_1$ or 
$F_2$: that is (after free cancellation if necessary) $g_i$ is a 
reduced word in $(X_1\cup X_1^{-1})^\ast$ or $(X_2\cup X_2^{-1})^\ast$
\end{comment}
Given an arbitrary word $g\in F_1\ast F_2$,  
we should like an algorithm to write $g$ in double coset normal form
with respect to some choice of double coset representatives. We may 
assume (after free cancellation if necessary) that $g$ is a 
reduced word in $((X_1\cup X_2)\cup ( X_1^{-1}\cup X_2^{-1}))^\ast$.
 
First we say $g$ is in \emph{reduced form} if $g = g_1 \cdots g_k$, where
$g_1 \in F_1$ or $g_1 \in F_2$ and 
for 
$k > 1$,    $g_i \in F_i \backslash H_i$, for $i=1$ or $2$ 
 and $g_i$, ${g_{i+1}}$ belong to different factors. As $F_1$ and 
$F_2$ have solvable subgroup membership problem we can rewrite $g$ in
reduced form. More precisely, we construct a Stallings automaton $A_i$ for
$H_i\le F_i$ and choose a maximal subtree $T_i$ of $A_i$. 
Then we write $g=f_1\cdots f_r$, in free product normal
form (each $f_i$ in a factor and successive terms from different factors).
Now using the appropriate Stallings automata we write $f_r=h_rs_r$, where
$h_r\in H_i$ and $s_r$ is a coset representative. (Read the maximal 
possible left prefix $w_r$ of $f_r$ in $A_i$. Then $h_r$ is the 
maximal subword of $w_r$ which belongs to $H_i$ and $s_r$ is the
unique word such that $f_r=h_rs_r$, reduced as written.)

Using the maps $\phi_1$ and $\phi_2$ we now write $h_r$ in terms of the
generators of the other factor, to obtain $h^\prime_r$ and then 
replace $f_{r-1}$ with $f_{r-1}h^\prime_r$. Now begin again with
$f_{r-1}$ instead of $f_r$. Continue till the word is exhausted.   
This gives $g=g_1\cdots g_t$ in reduced form.

To rewrite in double coset normal form we need to  
 rewrite
each $g_i$ in double coset normal form in its factor, 
with respect to some (fixed) set
of double coset representatives.  We shall show that the following algorithm does this.\\[1em]

\begin{comment}
${g = h_1(X_1)s_1h_2(X_2)t_2h_3(X_1) \cdots}$

$k=1 \Rightarrow$ $g \in F_1$ or $g \in F_2$



Let ${\Gamma_{H_1}}$ be a folded subgroup graph of $H_1$. Take two copies.

For some word $w = h_1sh_2$, the question is how to find $s$.\\[1em]
\end{comment}
\noindent{\bf Algorithm}:\\[1em]
Let $X_i=X$, $F_i=F$,  $H_i=H$ and  $A_i=A$.
Let $w$ be a reduced word in $F$. 
We say the word $w$ is {\em readable} by $A$ if, starting in the 
start state $1$ with input $w$, the automaton finishes with the empty
word in some arbitrary state. If $w$ is readable by $A$ and the finishing
state is $1$ then we say $w$ is {\em accepted} by $A$.  
  

Let $h$ be the maximal prefix of $w$ accepted by $A$; so $h\in H$.
Write $w=h\circ v$ (where the $\circ$ means ``freely reduced product''). 
Let $s_1$ be the maximal subword of $v$ that is readable in $A$ and let $v=s_1\circ u$. 
Let $\bar u$ be the inverse of $u$ (if $u=u_1\cdots u_t$ then $\bar u=
u_t^{-1}\cdots u_1^{-1}$) and let $h_2=\bar t$, 
where $t$ is the maximal prefix 
of $\bar u$ accepted by $A$. Let $\bar u= \bar h_2 \circ s$, let $r$ be the maximal prefix
of $s$ that is readable in $A$, and let $s_2=\bar r$. Now, there exists 
a unique reduced word $f\in F$ 
(possibly empty) such that $\bar u= \bar h_2\circ \bar s_2\circ \bar f$ and we
have 
$w=h_1s_1fs_2h_2$. The set of double coset representatives consists of all words $s_1fs_2$ constructed by
this algorithm. (Note that there is no choice involved in the process, so $s_1fs_2$ is uniquely determined
by $w$.) 

In fact the algorithm is simpler than the description above; there is no need for it to find $s_i$  it 
simply finds the maximal prefix $h_1$ of $w$  accepted  by $A$  and so $w=h_1\circ v$. Then it finds the
maximal prefix $p$ of $\bar v$ accepted by $A$. Setting $h_2=\bar p$ we now have $w=h_1\circ d \circ h_2$, for
some uniquely determined word $d$, which is the required double coset representative. 


\begin{example}
Picture required.
\begin{align*} 
g &= x_1^3x_2^2y_1y_2 \\ 
&= (x_1^2)(x_1x_2)(x_2^{-1})^{-1} \cdot (y_1)(y_2^{-1})^{-1}
\end{align*}
\end{example}

Now we write each syllable $g_i$ of $g=g_1\cdots g_t$ in normal form using the algorithm above. 
This gives $g_i=h_{i,1}d_ih_{i,2}$, with $d_i\in S_1\cup S_2$. Using $\phi_1$ or $\phi_2$, as appropriate,
we now write $h_{i,1}$ and $h_{i,2}$ as words in $F(Z)$. For $i=1,\ldots , t-1$, we reduce the 
word $h_{i,2}h_{(i+1),1}\in F(Z)$ to give a reduced word $h_i\in F(Z)$ and set $h_0=h_{1,1}$ and  $h_{t+1}=h_{t,2}$.
Then $g$ has normal form $h_0d_1h_1\cdots d_th_{t+1}$. 
\section{The generalised folding process}
We shall restrict to malnormal $H_i$ to make the generalised folding process terminate.
\begin{align*} 
G &= F_1 \underset{H_1=H_2}{\ast}F_2 \\ 
K &= \langle k_1, \ldots, k_t \rangle \leqslant G
\end{align*}
%
%
%%%%%%%%%%%%%%%%%%%%%%%%%%%%%%%%%%%%%%%%%%%%%%%%%
\bibliographystyle{plain}
\bibliography{membership}
\end{document}
