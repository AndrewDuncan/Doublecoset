\documentclass{beamer}
%
\usepackage{psfrag,comment}
%\usepackage{texpower}
%\usepackage[colormath]{texpower}
%
\mode<presentation>
{
  %\usetheme{Malmoe}
%\usetheme{Montpellier}
\usetheme{Pittsburgh}
  % or ...

  \setbeamercovered{transparent}
  % or whatever (possibly just delete it)
}

\title%[Short Paper Title] % (optional, use only with long paper titles)
{Automorphisms of partially commutative groups}

\author%[Author, Another]% (optional, use only with lots of authors)
{Andrew Duncan \\[1em]
(joint work with  V N Remeslennikov)}
\date{October  11th, 2010}
\newcommand{\la}{\langle}
\newcommand{\ra}{\rangle}
\newcommand{\abs}[1]{\left|\mathinner{#1}\right|}
\newcommand\eps{\varepsilon}
\newcommand{\Oh}{\mathcal{O}}
\def\FF{{\mathbb F}}
\def\NN{{\mathbb N}}
\def\QQ{{\mathbb Q}}
\def\ZZ{{\mathbb Z}}
\def\CC{{\mathbb C}}
\def\RR{{\mathbb R}}
\def\nul{\emptyset }
\def\D{\Delta }
\def\b{\beta }
\def\k{\kappa }
\def\r{\rho }
\def\x{\chi }
\def\e{\varepsilon }
\def\G{\Gamma }
\def\g{\gamma }
\def\a{\alpha }
\def\t{\tau }
\def\s{\sigma }
\def\S{\Sigma }
\def\w{\omega }
\def\pd{\partial}
\def\fC{{\textswab C}}
\newcommand{\FR}{\operatorname{FR}}
%\newcommand{\lk}{\operatorname{lk}}
\newcommand{\ad}{{\operatorname{ad}}}
\newcommand{\cd}{\textrm{cdim}}
\newcommand{\CD}{\textrm{CD}}
\newcommand{\CT}{\textrm{CT}}
\newcommand{\CSA}{\textrm{CSA}}
\newcommand{\cL}{{\cal L}}
\newcommand{\cLg}{{\cal L}^{\textrm{gp}}}
\newcommand{\cLp}{{\cal L}^{\textrm{pre}}}
\newcommand{\cG}{{\cal G}}
\newcommand{\cF}{{\cal F}}
\newcommand{\cP}{{\cal P}}
\newcommand{\sdc}{>}
\newcommand{\sac}{<}
\newcommand{\nsdc}{\ngtr}
\newcommand{\nsac}{\nless}
\newcommand{\edc}{\ge}
\newcommand{\eac}{\le}
\newcommand{\nedc}{\ngeq}
\newcommand{\neac}{\nleq}
\newcommand{\Aut}{\operatorname{Aut}}
\newcommand{\Out}{\operatorname{Out}}
\newcommand{\cl}{\operatorname{cl}}
\newcommand{\icl}{\operatorname{icl}}
\newcommand{\acl}{\operatorname{acl}}
\newcommand{\fcl}{\operatorname{fcl}}
\newcommand{\CS}{\operatorname{CS}}
\newcommand{\maps}{\rightarrow}
\newcommand{\ov}[1]{\overline{#1}}
\newcommand{\bs}{\backslash}
\newcommand{\conj}{\operatorname{conj}}
\newcommand{\St}{\operatorname{St}}
\newcommand{\cSt}{\St^{\conj}}
\newcommand{\LInn}{\operatorname{LInn}}
\newcommand{\Inn}{\operatorname{Inn}}
\newcommand{\rInn}{\ov{\LInn}}
\newcommand{\Tr}{\operatorname{Tr}}
\newcommand{\Trp}{\operatorname{Tr}_\perp}
\newcommand{\tr}{\operatorname{tr}}
\newcommand{\kSt}{\St_{K}}
\newcommand{\ckSt}{\St^{\conj}}
\newcommand{\Conj}{\operatorname{Conj}}
\newcommand{\oConj}{\Conj^{*}}
\newcommand{\iConj}{\Conj_{\operatorname{I}}}
\newcommand{\AConj}{\Conj_{\operatorname{A}}}
\newcommand{\CConj}{\Conj_{\operatorname{C}}}
\newcommand{\NConj}{\Conj_{\operatorname{N}}}
\newcommand{\RConj}{\Conj_{\operatorname{R}}}
\newcommand{\VConj}{\Conj_{\operatorname{V}}}
\newcommand{\aOonj}{\overline{\aConj}}
\newcommand{\sConj}{\Conj_{\operatorname{S}}}
\newcommand{\Isol}{\operatorname{Dom}}
\newcommand{\cls}{\cl^\sharp}
\newcommand{\cmp}{{\operatorname{{comp}}}}
\newcommand{\GL}{\operatorname{GL}}
\newcommand{\lk}{\operatorname{lk}}
\newcommand{\IFF}{if and only if } %
%
\newcommand{\be}{\begin{enumerate}}
\newcommand{\ee}{\end{enumerate}}
\newcommand{\bei}{\begin{itemize}}
\newcommand{\eei}{\end{itemize}}
\newcommand{\bd}{\begin{description}}
\newcommand{\ed}{\end{description}}
%%%%%%%%%%%%%%
%%%%%%%%%%%%%%%%%%%%%%
\newcommand\RAS[2]{\overset{#1}{\underset{#2}{\Longrightarrow}}}
\newcommand\LAS[2]{\overset{#1}{\underset{#2}{\Longleftarrow}}}
\newcommand\DAS[2]{\overset{#1}{\underset{#2}{\Longleftrightarrow}}}
\newcommand\OUTS[5]{#1
  \overset{#2}{\underset{#3}{\Longleftarrow}} #4
  \overset{#2}{\underset{#3}{\Longrightarrow}} #5}
\newcommand\INS[5]{#1
  \overset{#2}{\underset{#3}{\Longrightarrow}} #4
  \overset{#2}{\underset{#3}{\Longleftarrow}} #5}
%%%%%%%%%%%%%%%%%%%%%%
\newcommand\RA[1]{{\underset{#1}{\Longrightarrow}}}
\newcommand\LA[1]{{\underset{#1}{\Longleftarrow}}}
\newcommand\DA[1]{{\underset{#1}{\Longleftrightarrow}}}
\newcommand\OUT[4]{#1
 {\underset{#2}{\Longleftarrow}} #3
 {\underset{#2}{\Longrightarrow}} #4}
\newcommand\IN[4]{#1
 {\underset{#2}{\Longrightarrow}} #3
 {\underset{#2}{\Longleftarrow}} #4}
%%%%%%%%%%%%%%%%%%%%
%%%%%%%%%%%%%%%%%
%\newtheorem{theorem}{{\bf Theorem}}[section]
%\newtheorem{corollary}[theorem]{{\bf Corollary}}
%\newtheorem{definition}[theorem]{{\bf Definition}}
%%%%%\newtheorem{example}[theorem]{{\bf Example}}
%\newtheorem{lemma}[theorem]{{\bf Lemma}}
\newtheorem{proposition}[theorem]{{Proposition}}
\newtheorem{remark}[theorem]{{Remark}}
\newtheorem{delusion}[theorem]{{Question}}

%%%%%%%%%%%%%
\begin{document}
%

\begin{frame}
  \titlepage
\end{frame}

\begin{frame}
  \frametitle{Outline}
  \tableofcontents[pausesections]
  % You might wish to add the option [pausesections]
\end{frame}
\section{Introduction}
\begin{frame}
  \frametitle{}
The class of partially commutative groups (RAAGs, graph groups)
contains finitely generated free groups and f.g. Abelian groups
and is closed under free and direct products.\\[1em]\pause
Automorphisms groups of p.c. groups: contain $\Aut(\FF_n)$ and 
$\GL_n(\ZZ)$ \pause and automorphism groups of free and direct products of ...
\\[1em]
\pause

Given a partially commutative group $G$ this talk
\bei
\item describes many important subgroups of $\Aut(G)$;\pause
\item states decomposition theorems for $\Aut(G)$ (c.f. decomposition theorems
for linear groups);\pause
\item and gives a new projection theorem onto the automorphism groups
of smaller p.c. groups (c.f. restriction and projection maps of Charney and Vogtmann).
\eei
\end{frame}
%
%%%%%%%%%%%%%%%
\begin{frame}
\frametitle{Partially Commutative Groups}
 $\G$ a graph, with vertices $X=\{x_1,\ldots ,x_n\}$ 
and edges $E$ a set of $2$-subsets of $X$.\\[1em]
\pause
The {\em partially commutative group} $G=G(\G)$ is 
\[\la X\,|\, [x,y]=1, \forall \{x,y\}\in E\}\ra.\]
\pause
(semi-free, graph, right-angled, trace, locally free)\\\pause
e.g. \\
free groups: $\G$ the null graph\\\pause
free Abelian groups: $\G$ the complete graph\\[1em]\pause
$\FF_2\times \FF_2$:~~~~~~~\raisebox{-.5cm}{%\rule[-1cm]{0pt}{2cm}
\includegraphics<6->[scale=0.3]{square.eps}}\pause
~~~~~~~~~~~~~~$\ZZ^2*\ZZ$:~~~~ \includegraphics<7->[scale=0.3]{line.eps} 
\end{frame}
%
%%%%%%%%%%%%%%%
\section{Automorphisms}
\begin{frame}
  \frametitle{Automorphisms}
\begin{itemize}
\item Two graph groups $G(\G_1)$ and $G(\G_2)$ are isomorphic if and only if $\G_1\cong \G_2$.
\pause
\item Generators for $\Aut(G)$ are known (Laurence, J.London Math. Soc. '95 , using results of 
Servatius '89).\pause
\item Decomposing $\Aut(G)$ into two parts, one which embeds in $\Aut(G^{\textrm{ab}})$ and
the other for which peak reduction works, Day obtains a presentation for $\Aut(G)$.\pause 
\item 
Charney, Vogtmann and Crisp have shown that $\Out(G)$ is virtually torsion
free, and has finite virtual cohomological dimension.\pause ~
For $\G$ with no triangles,  a finite dimensional, contractible space $O(G)$ 
on which $\Out(G)$ acts properly is also found.   (Geom Topol. '07)\pause
\item Gutierrez, Piggot and Ruane describe a semidirect product decomposition
$(\Inn(G)\rtimes A_1)\rtimes A_2$ of $\Aut(G)$ in the more general setting 
where $G$ is a graph product of f.g. Abelian groups.
\end{itemize}
\end{frame}
%%%%%%%%%%%%%%%
\begin{frame}
  \frametitle{Automorphisms cont. }
\begin{itemize}
\item D, Kazachkov and Remeslennikov have described the structure of certain subgroups of 
$\Aut(G)$ and show that the stabiliser of the lattice $L$ is an arithmetic subgroup of
$\Aut(G)$ (GGD '10).\pause
\item Let $\a:\Aut(G)\maps \Aut(G^{\textrm{ab}})$ be the canonical map. Noskov shows 
that Im$(\a)$ is an arithmetic subgroup of $\GL_n(\RR)$,  and also that, for some
$\G$, $\Aut(G)$ does not have property $T$.  
\end{itemize}
\end{frame}
%%%%%%%%%%%%%%%%%%%%%%
\section{Structure of the automorphism group}

\begin{frame}
\frametitle{Reduction to connected graphs }
Let $\G$ have connected components $\G_1,\ldots \G_n$, where 
$\G_i$ has vertex set $X_i$, and let $G_i=\la X_i\ra$.  Then $G_i=G(\G_i)$ 
and
\[G=G_1\ast \cdots \ast G_n.\]

\pause 
\bei
\item
Fouxe-Rabinovitch gave a presentation for $\Aut(G)$ in terms of
presentations of $\Aut(G_i)$.\pause
\item
 Collins and Gilbert refine this presentation and \pause 
\item
prove that if a peak reduction 
theorem holds in  $\Aut(G_i)$, for all $i$, then the same is true of $\Aut(G)$. 
\eei
\end{frame}
%%%%%%%%%%%%%%%%%%%%%%
\begin{frame}
\frametitle{Isolated vertices}
As before: $\G$ has connected components $\G_1,\ldots \G_n$, where 
$\G_i$ has vertex set $X_i$, and   $G_i=\la X_i\ra$.\\[1em]\pause

If $\G$ has isolated vertices then $G=G_1\ast \cdots G_m\ast \FF_k$.\\[1em] \pause

In this case the description of $\Aut(G)$ using Collins \& Gilbert is technical: there
is no obvious semi-direct product decomposition\\[1em]\pause
 the existence
of isolated vertices is one major obstruction to a clear structural description
of $\Aut(G)$.\\[1em]\pause

If $\G$ has no isolated vertices a natural structure theorem exists. 
\end{frame}
%%%%%%%%%%%%%%%%%%%%%%
\begin{frame}
\frametitle{No isolated vertices}

For $i\neq j$ and $x\in X_i$ there is an automorphism $\phi$ of $G$ given by
\[y\phi =y^x, \textrm{ for all } y\in X_j \textrm{ and } y\phi = y, 
\textrm{ otherwise}.\]
Let $\FR(G)$ be the subgroup of $\Aut(G)$ generated by all such automorphisms.

\pause
\begin{theorem}[Collins \& Gilbert]
\label{theorem:FRker}
Suppose that no component of $\G$ is an isolated vertex. 
Define $\bar G=G_1\times \cdots \times G_n$. 
Then 
$\FR(G)$ is the kernel of the
canonical map from $\Aut(G)$ to $\Aut(\bar G)$. Moreover $\FR(G)$ has a normal 
series 
\[
1<P_{n-1}<\cdots <P_2<\FR(G)
\]
such that, setting $\FR_i(G)=\FR(G)/P_i$, 
\be[(a)]
\item $\FR(G)=P_i \rtimes \FR_i(G)$, 
\item $\FR_i(G)=\FR(G_1\ast \cdots \ast G_i)$ and
\item all the $P_i$ are finitely generated.
\ee
\end{theorem}
\end{frame}
%%%%%%%%%%%%%%%%%%%%%%
\begin{frame}
\frametitle{Servatius-Laurence generators}
{\textcolor{red}{From now on assume that $\G$ is a connected graph.
}}
\pause
Servatius and Laurence defined the following sets of automorphisms.
\begin{itemize}
\item The set $V$ of graph automorphisms. Given a graph automorphism
 $\theta$ of $\G$ 
there is a group automorphism $\phi$ of $G$  such that $x\phi =x\theta$, for 
all $x\in X$. 
\item Inversions. $I=\{\phi\in \Aut(G): x\phi=x^{\pm 1},$ for all $x\in X\}$.
\item Transvections. $T=\{\phi\in \Aut(G):$ there exist $x,y\in X^{\pm 1}$ 
such that $x\phi=xy$ and $z\phi=z$ for all $z\in X\backslash x\}$. 
\item Basis conjugating automorphisms. $C=\{\phi\in \Aut(G):$ for all
$x\in X$ there exists $g_x\in G$ such that $x\phi=x^{g_x}\}$. 
\item Elementary basis conjugating automorphisms. $B=\{\phi\in C:$ for some 
$y\in X$ and fixed component $Y$ of $\G\bs  \{y^\perp \}$,   $x\phi =x^y$ if $x\in Y$ and 
$x\phi=x$ otherwise$\}$. 
\end{itemize}
\pause
\begin{theorem}[Laurence] $C=\la B\ra$ and $\Aut(G)=\la V\cup I\cup T\cup B\ra$.
\end{theorem}
\end{frame}
%%%%%%%%%%%%%%%%%%%%%%
\begin{frame}
\frametitle{Decomposition using graph isomorphisms}
The set $V$ of graph automorphisms: is a finite subgroup of $\Aut(G)$, \pause and 
we'd like to decompose the group $\Aut(G)$ as \\$\la$ something $\ra \la V\ra$, \pause
nicely.

\end{frame}
%%%%%%%%%%%%%%%%%%%%%%
\begin{frame}
\frametitle{Compression of $\G$}
For $x\in X$ define 
\bei
\item the \em{star} $x^\perp$ of $x$ to be 
\[x^\perp=\{y\in X: x\textrm{ and }y\textrm{ are incident vertices}\}\] \pause 
\item
the \em{link} of $\lk(x)$ to be 
\[\lk(x)=x^\perp\bs x,\] 
\pause
\item
an equivalence relation $\sim$ on $X$ by \\
\centerline{$x\sim y$ if 
$x^\perp=y^\perp$ or $x^\perp\bs x=y^\perp\bs y$. }
~\\
\eei

\pause
{\bf Compression. } 
Form a new graph $\bar\G$ from $\G$ by identifying equivalence classes
of $\sim$.
 \end{frame}
%%%%%%%%%%%%%%%%%%%%%%
\begin{frame}
\frametitle{Decomposition over graph automorphisms}
 Let $\Aut({\bar \G})$ be the group of vertex automorphisms
of $G(\bar \G)$ which ``respect equivalence classes of $\sim$''.\\[1em]
\pause
Recall:\\
$I=$ inversions\\
$T=$ transvections\\
$B=$ elementary basis conjugating automorphisms\\ 
\pause
\begin{theorem}
\[\Aut(G)\cong \la I\cup T\cup B\ra\rtimes \Aut(\bar \G).\] 
\end{theorem}
\pause
It remains to describe the structure of $\la I\cup T\cup B\ra$.\\[1em]
\pause
Define $\Aut^*(G)=\la I\cup T\cup B\ra$.
\end{frame}
%%%%%%%%%%%%%%%%%%%%%%
\begin{frame}
\frametitle{The Charney-Crisp-Vogtmann decomposition}
Charney, Crisp and Vogtmann consider the subsets 
\[J_x=[x]\cup x^\perp\]
of $X$,  and subgroups $V_x=\la J_x\ra$.\\[1em]
\pause

A vertex is {\bf biggest} if $\lk(x)\subseteq y^\perp$ implies $\lk(y)\subseteq x^\perp$.\\[1em]
\pause 
There exists a homomorphism 
%\[R_x:\Out^*(G)\maps \Out^*(V_x),\]
%for all maximal $x\in X$, \pause and a homomorphism
\[R:\Out^*(G)\maps \prod\Out^*(V_x),\]
where the product is over all biggest $x$. 

\pause
\begin{theorem}[Charney-Vogtmann-Crisp]
The kernel of %$\prod R_x$ 
$R$ is a finitely generated free Abelian subgroup of $\Conj(G)$.
\end{theorem}
\pause
A consequence is that $\Out(G)$ has finite virtual cohomological dimension.
\end{frame}
%%%%%%%%%%%%%%%%%%%%%%
\begin{frame}
\frametitle{Another decomposition: the sets K}
For $Y\subseteq X$ define \[Y^\perp={\displaystyle{\bigcap_{y\in Y}}}\, y^\perp.\]\pause
For $x\in X$ define the set 
\[\ad(x)=(\lk(x))^\perp.\] 
\pause

Let $K=\{\ad(x):x\in X\}$. 

\pause
\begin{itemize}
\item $x\in \ad(x)$, for all $x\in X$.
\item If $y\in \ad(x)$ then $\ad(y)\subseteq \ad(x)$. 
\end{itemize}
\pause

$K$ $\leadsto$ two graphs: \pause
\bei
\item the compressed graph $\bar G$; $\ad(x)$ is incident to $\ad(y)$ iff $[x,y]=1$ \pause and  
\item the directed transvection graph; $\ad(x)$ is incident to $\ad(y)$ iff 
$\ad(x)\subseteq \ad(y)$.
\eei
\pause

Let $G_x$ denote the (partially commutative) group $\la \ad(x)\ra$. 

Then $G_x=C_G(\lk(x))$. 
\end{frame}
%%%%%%%%%%%%%%%%%%%%%%
\begin{frame}
\frametitle{Stabilisers}
Let $\Aut^*(G)=\la I\cup T\cup B\ra$ and, 
for $x\in X$, define 
\begin{itemize}
\item $\St_K=\{\phi \in \Aut^*:G_x\phi =G_x, \forall x\in X\}$;\pause
\item $\cSt_K=\{\phi \in \Aut^*:G_x\phi=G_x^{g_x}, \forall x\in X\}$.\pause
\end{itemize}


\begin{theorem}
$\Aut^*(G)=\cSt_K$ and $\St_K=\la I\cup T\ra$.
\end{theorem}
\end{frame}
%%%%%%%%%%%%%%%%%%%%%%
\begin{frame}
\frametitle{Basis conjugating automorphisms}
$\Conj=\la B\ra$ the group of basis conjugating automorphisms.\pause
\bei
\item
$\a_{Y,y}$ denotes the element of $B$ such that $x\mapsto x^y$ if 
$x\in Y$ and $x\mapsto x$, otherwise.\pause
\item
$\tr_{y,z}$ denotes the element of $T$ such that $y\mapsto yz$, and 
$a\mapsto a$,  if $a\neq y^{\pm 1}$.  
\eei  
\pause

Define  
\bei 
\item   
$\NConj=\{\phi\in \Conj : \forall x\in X, \exists f_x\in G, \textrm{
 such that } \forall y\in \ad(x), y\phi=y^{f_x} \}.$ \pause
\item $\sConj=\St_K\cap \Conj$  
\eei 
\pause

\begin{theorem}
\be
\item $\sConj$ is generated by $\a_{Y,y}$ such that $|Y|={z}$, for some 
$z\in X$. \pause In this case  $\a_{Y,y}=\tr_{z^{-1},y}\tr_{z,y}$.\pause
\item 
$\Inn\lhd \NConj\lhd \Aut^*$.
\ee
\end{theorem}
\begin{delusion}
Is 
$\cSt_K=\Conj_N\cdot \St_K$?
\end{delusion}
\end{frame}
%%%%%%%%%%%%%%%%%%%%%%
\begin{frame}
\frametitle{Example}
\begin{center}
\psfrag{a}{$a$}
\psfrag{b}{$b$}
\psfrag{c}{$c$}
\psfrag{r}{$r$}
\psfrag{s}{$s$}
\psfrag{t}{$t$}
\psfrag{v}{$v$}
\includegraphics[scale=0.4]{diagex.eps}
\end{center}
Let $C=\{a,r,s\}$ and 
 $\a=\a_{C,v}$, $\t=\tr_{v,a}\tr_{v,b}\tr_{v,a^{-1}}$. Set 
$\phi=\a\t$. 
\begin{equation*}
z\phi=
\left\{
\begin{array}{ll}
z, & \textrm{ if } z=b, c, t\\
vb^a, &\textrm{ if } z=v\\
z^{vb^a} & \textrm{ if } z\in C
\end{array}
\right.
\end{equation*}
$\phi$ cannot be written as $\g\delta$, where $\delta\in \Conj$
and $\g\in St_K$.
\end{frame}
%%%%%%%%%%%%%%%%%%%%%%
\begin{frame}
\frametitle{Graphs free of domination}
%\begin{defn}\label{def:isols}
If $x$ and $y$ are vertices of $X$ such that 
$[x,y]\neq 1$ and
$\ad(x)\subseteq \ad(y)$ \pause

(iff $d(x,y)=2$ and $y^\perp\bs y\subseteq x^\perp$)

\pause

then we say that 
$x$ {\em dominates} $y$. The set of all vertices dominated by $x$ is denoted 
$\Isol(x)=\{u\in X\,|\, x \textrm{ dominates } u\}$. The set of all dominated
vertices is denoted $\Isol(\G)=\cup_{x\in X}\Isol(x)$. \\[1em]
%\end{defn}

\pause

  \begin{theorem}\label{theorem:nolocisol} Let $\G$ be a connected graph 
such that $\Isol(\G)=\nul$. Then 
\be
\item 
$\sConj(G)=\{1\}$;
\item $\Conj(G)=\NConj(G)$, so is normal in $\cSt_K$; and  
\item $\cSt_K=\Conj(G)\cdot \St_K$; so
\ee
\[\cSt_K=\Conj(G)\rtimes \St_K.\]
\end{theorem}
\end{frame}
%%%%%%%%%%%%%%%%%%%%%%
\begin{frame}
\frametitle{Structure of $\St_K$}
$K$ is ordered by inclusion and the {\bf height} $h(x)$ of  $x\in X$ is the
maximal length of a strictly ascending  chain $\ad(x_0)\subset\cdots \subset \ad(x_h)=\ad(x)$.\\[1em]
\pause

$K(r)$ denotes the elements of height $r$ and $K^{\max}$ those of maximal height. 
\pause

The building blocks of $\St_K$ are obtained by restricting automorphisms to levels:\\[1em]
Let $h$ be the maximal height of an element of $X$.\\[1em] 
For $\phi\in \St_K$, $x\in X$ and $r$ such that $0\le r\le h$ let $\phi_{r,x}$ be the 
map given by
\bei
\item $y\phi_{r,x}=y\phi$, if $\ad(y)\subseteq \ad(x)$ and $h(y)=r$;
\item $y\phi_{r,x}=y$, otherwise.
\eei
  
\pause
$\phi_{r,x}$ extends to an element of $\St_K$. 
\end{frame}
%%%%%%%%%%%%%%%%%%%%%%
\begin{frame}
\frametitle{}
$\St_{x,r}$ is the subgroup consisting of all elements $\phi_{x,r}$.
\pause
\begin{theorem}\label{thm:stkdecomp}
 Let $h$ be the maximal height of an element of $X$. Then 
\be
\item \[\St_{r}=\prod_{x\in V(r)^{\cmp}} \St_{x,k}\] is a subgroup of $\St_K$, 
for $r=0,\ldots ,h$ \pause and 
\item 
\begin{gather*}
\St(K)=\St_{h}\rtimes (\St_{h-1} \rtimes 
\cdots \rtimes (\St_{1}\rtimes \St_{0})\cdots )
\end{gather*}
\ee
\end{theorem}
\pause
\begin{theorem}\label{prop:stx1gen}
Given $y\in [x]$ and $z\in \ad(x)$ there exists 
a transvection $y\mapsto yz$ in $T$. Let $T_x$ denote the set
of all such transvections. Then 
$\St_{x,h}$ is generated by 
\be[(i)]
\item
$T_x$ and 
\item $I_x$,  the set of elements $y\mapsto y^{-1}$ in $I$ such that $y\in [x]$.
\ee
\end{theorem}
\end{frame}
%%%%%%%%%%%%%%%%%%%%%%
\begin{frame}
\frametitle{Relationship to Charney-Crisp-Vogtmann}
There exists a homomorphism 
\[S:\Out^*(G)\maps \prod\Out^*(G_x),\]
where the product is over a set of representatives of maximal elements
of $X$. \\[1em]
\pause

\bei
\item $\ker(S)=\NConj(G)$;
\item $\NConj(G)$ contains  a free Abelian subgroup $\AConj(G)$ equal to the kernel of $R$.
\item $\AConj(G)$ is of rank $\sum_{x\in X} c(x)-1$, where $c(x)$ is the number of components
of $\G\bs\{x\}$. 
\eei
\end{frame}
\end{document}