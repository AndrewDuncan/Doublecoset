\documentclass[a4paper,12pt]{article}
\usepackage{amsmath,amsfonts,amsthm,amscd,amssymb,latexsym}%,eufrak}
%%%%%%%%%%%%%
\usepackage{caption}
\usepackage{subcaption}
\usepackage{enumerate,graphicx,psfrag}%,subfigure}%,jchangebar,oldgerm}
\usepackage[mathscr]{eucal}
\usepackage[usenames]{color}
\usepackage{url}
\usepackage[shortlabels]{enumitem}
\usepackage{comment}
%\usepackage[utf8]{inputenc}
\usepackage[T1]{fontenc}
\usepackage{showkeys}
%%%%%%%%%
\sloppy

%%\documentclass[12]{article}


%%%%%%%%%%%%%%%%%%%%%%%
%
%
%
\title{The subgroup membership problem
}
\author{Andrew J. Duncan, Elizaveta Frenkel}

\renewcommand{\a}{\alpha }
\renewcommand{\b}{\beta }
\newcommand{\G}{\Gamma }
\newcommand{\g}{\gamma }
\newcommand{\D}{\Delta }
\renewcommand{\d}{\delta }
%\def\vd{\vardelta}
\newcommand{\ep}{\epsilon }
\newcommand{\e}{\varepsilon }
\newcommand{\z}{\zeta }
%\eta
\renewcommand{\th}{\theta }
\newcommand{\T}{\Theta }
\renewcommand{\i}{\iota }
\renewcommand{\k}{\kappa }
\renewcommand{\l}{\lambda }
\renewcommand{\L}{\Lambda }
%\mu
%\nu
%\xi
%omicron
%\pi
\renewcommand{\r}{\rho }
\newcommand{\s}{\sigma }
\renewcommand{\S}{\Sigma }
\renewcommand{\t}{\tau }
\newcommand{\up}{\upsilon }
\newcommand{\U}{\Upsilon }
%\phi
\newcommand{\x}{\chi }
%\psi
\newcommand{\W}{\Omega }
\newcommand{\w}{\omega }
%%%%%%%%%%%%%%%%%%%%%%%%%%%%%%%
%%%%%%%%%%%%%%%%%%%%%%%%%%%%%
\newcommand{\pd}{\partial}
\newcommand{\wht}{\widehat}
%\newcommand{\cC}{{\mathcal C}}
%\newcommand{\cdim}{\texttt{cdim}}
\newcommand{\fC}{{\textswab C}}
\newenvironment{ef}{\noindent\color{blue} \bf EF: }{}
%
\newcommand{\cA}{{\cal{A}}}
\newcommand{\cD}{{\cal{D}}}
\newcommand{\cF}{{\cal{F}}}
\newcommand{\cH}{{\cal{H}}}
\newcommand{\cJ}{{\cal{J}}}

\newcommand{\cK}{{\cal{K}}}
\newcommand{\cP}{{\cal{P}}}
\newcommand{\cR}{{\cal{R}}}
\newcommand{\cS}{{\cal{S}}}
\newcommand{\cW}{{\cal{W}}}
\newcommand{\cQ}{{\cal{Q}}}
%\newcommand{\GG}{\ensuremath{\mathbb{G}}}
\newcommand{\pp}{\mathbf{p}}
%%%%%%%%%%%%%%%%%%%%%%%%%%%%%%
\newcommand{\nul}{\emptyset }

%%%%%%%%%%%%%%%%%%%%%%%%%%%%%%
\newtheorem{theorem}{Theorem}[section]
\newtheorem{lemma}[theorem]{Lemma}
\newtheorem{corollary}[theorem]{Corollary}
\newtheorem{proposition}[theorem]{Proposition}
\newtheorem{axiom}[theorem]{Axiom}
\newtheorem{definition}[theorem]{Definition}
\newtheorem*{defn*}{Definition}
\newtheorem{conjecture}[theorem]{Conjecture}
%cvs -d :pserver:najd2@cvs.mas.ncl.ac.uk:/CVS/najd2
\newtheorem{exam}[theorem]{Example}
%\newtheorem{comment}[theorem]{Comment}
%
%
\newenvironment{example}{\begin{exam} \rm}{\end{exam}}
%
%
%
\newtheorem{remk}[theorem]{Remark}
\newenvironment{remark}{\begin{remk} \rm}{\end{remk}}
%
%%%%%%%%%%%%
\numberwithin{equation}{section}
\numberwithin{figure}{section}
%%%%%%%%%%%%%%%%%%%%
\newcommand{\Loop}{\operatorname{Loop}}
\newcommand{\Iso}{\operatorname{Isom}}
\newcommand{\Aut}{\operatorname{Aut}}
%%%%%%%%%%%%%%%%%%%
\renewcommand{\AA}{\ensuremath{\mathbb{A}}}
\newcommand{\ZZ}{\ensuremath{\mathbb{Z}}}
\newcommand{\QQ}{\ensuremath{\mathbb{Q}}}
\newcommand{\RR}{\ensuremath{\mathbb{R}}}
\newcommand{\NN}{\ensuremath{\mathbb{N}}}
\newcommand{\CC}{\ensuremath{\mathbb{C}}}
\newcommand{\FF}{\ensuremath{\mathbb{F}}}
%\renewcommand{\ker}{\verb"Ker"}
\newcommand{\cC}{\mathcal{C}}
\renewcommand{\cF}{\mathcal{F}}
\newcommand{\cO}{\mathcal{O}}
\renewcommand{\cS}{\mathcal{S}}
\newcommand{\la}{\langle}
\newcommand{\ra}{\rangle}
%\newcommand{\BA}{\ensuremath{\mathbb{A}}}
%%%%%%%%%%%%%%%%%%%%%%%%%%%%%%%%%%%%%%
\newcommand{\maps}{\rightarrow}
\newcommand{\ov}[1]{\overline{#1}}
\newcommand{\bs}{\backslash}
%%%%%%%%%%%%%%%%%%%%%%%%%%%%%%%
\newcommand{\be}{\begin{enumerate}}
\newcommand{\ee}{\end{enumerate}}
\newcommand{\bd}{\begin{description}}
\newcommand{\ed}{\end{description}}
\newcommand{\biz}{\begin{itemize}}
\newcommand{\eiz}{\end{itemize}}
%%%%%%%%%%%%%%%%%%%%%%%%%%%%%%%%%%%
%
\newenvironment{ajd}{\noindent\color{red} AJD }{}
\newenvironment{xxx}{\noindent\color{red}}{}
%
\begin{document}
\section{Summary of algorithm I}
\subsection{Preproccessing.}
\noindent\textbf{Input.}
\biz
\item generators $X$ of a free group $F(X)$.
\item A finite set $Y$ of elements of $F(X)$.
\eiz
\noindent\textbf{Output.}
\biz
\item 
$\G_A$, the Stallings automaton for the subgroup $H=\la Y\ra$ of $F(X)$.
\item
$L_T$, the set of words corresponding to a  maximal subtree $T$ of $\G_A$.
\item The set $P$ of non-diagonal elements of $V( \G_A\times \G_A)$ 
partitioned into equivalence classes of $\sim$ (i.e. vertices in the
same connected components of $\G_A\times \G_A$).
\item $P_0$ the set of representatives of equivalence classes of 
elements of $P$.
\item $C$ the set of connecting elements, one for each element of $P$.
\eiz
\noindent\textbf{Process.}
An upper bound for each step is given in brackets.
Let $N=\sum{y\in Y} |y|$.
\be
\item Construct $\G_A$. ($t_1(N)$.)
\item Construct a spanning tree $T$ for $\G_A$ and simultaneously
compute $L_T$. (This can easily be done if $T$ is constructed by
BFS or DFS, so one problem I was worrying about disappears.) ($O(N^2)$.)
\item Construct $\G_A\times \G_A$. ($t_3(N)$).
\item Find connected components of $\G_A\times \G_A$. This can be
done by doing a BFS or DFS of $\G_A\times \G_A$, which constructs
a spanning forest at the same time as finding the connected components.
\ee
\noindent\textbf{Question.} Is it better for our algorithm
to construct $T$ by BFS or DFS? Does it make any difference to the 
Algorithm II? 
\subsection{Algorithm I}
\noindent\textbf{Input.}
\biz
\item generators $X$ of a free group $F(X)$.
\item $w$, a reduced word in $F(X)$.
\item A finite set $Y$ of elements of $F(X)$
\item $\G_A$, the Stallings automaton for the subgroup $H=\la Y\ra$ of $F(X)$.
\item $P_0$, the set of representatives of equivalence classes of 
elements of $P$, the non-diagonal elements of $\G_A\times \G_A$.
\eiz
\end{document}